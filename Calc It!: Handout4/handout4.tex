
\title{Derivative Properties}
\author{Jonathan Kong}
\date{}
\documentclass[11pt]{scrartcl}
\usepackage{subfiles}
\usepackage[sexy]{evan}
\usepackage[utf8]{inputenc}
\usepackage{ upgreek }
\usepackage{geometry}
\geometry{%}
  letterpaper,
  lmargin=1.5cm,
  rmargin=1.5cm,
  tmargin=2 cm,
  bmargin=2cm,
  footskip=12pt,
  headheight=13.6pt}
\usepackage{url}
\urlstyle{tt}
\usepackage{float}
\usepackage{verbatim}
\usepackage[margin=1in]{geometry}
\usepackage{amsmath}
\usepackage{tcolorbox}
\usepackage[dvipsnames]{xcolor}
\usepackage{amssymb}\usepackage{dcolumn}
\newcolumntype{2}{D{.}{}{2.0}}
\begin{document}
\maketitle
\noindent

\section{The basic properties}
\noindent
If we had to use the limit definition of the derivative every time we wanted to find the derivative of a function, it would take extremely long. Instead, there are properties that we can use to make the process much quicker. Let's begin with the sum/difference property:\\
\noindent\\
Let $f$ and $g$ be differentiable functions. Then, $(f+g)'=f'+g'$.\\
\noindent\\
We can use the definition of derivative to show this:
\begin{align*}
(f+g)'(x) &= \lim_{h \to 0} \frac{f(x+h)+g(x+h)-f(x)-g(x)}{h}\\
          &= \lim_{h \to 0}\left(\frac{f(x+h)-f(x)}{h}+\frac{g(x+h)-g(x)}{h}\right)\\
          &= \lim_{h \to 0} \frac{f(x+h)-f(x)}{h}+\lim_{h \to 0} \frac{g(x+h)-g(x)}{h}\\
          &= f'(x)+g'(x)   
\end{align*}
\noindent
In a very similar manner, we can prove the constant property:\\
\noindent\\
Let $f$ be a differentiable function, and $c$ a real number. Then,  $(cf)'=c(f')$.\\
\noindent\\
We can show this by using the definition of derivative and also the constant property of limits:
\begin{align*}
    (cf(x))' & = \lim_{h \to 0} \frac{cf(x+h)-cf(x)}{h}\\
& = c\left(\lim_{h \to 0} \frac{f(x+h)-f(x)}{h}\right)\\
& = c(f'(x))
\end{align*}
\noindent
Next, we will take a look at the power rule:\\
\noindent\\
Let $n$ be any real number. If $f(x)=x^n$, then $f'(x)=nx^{n-1}$.\\
\noindent\\
The following proof is only for cases where $n$ is a positive integer. To prove that the power rule is true with $n$ being any real number, we must still consider other cases. 
$$f'(x)=\lim_{h \to 0} \frac{(x+h)^n-x^n}{h}$$
We can use the binomial theorem to expand out:
$$f'(x)=\lim_{h \to 0} \frac{\left(x^n+{n \choose 1}x^{n-1}h+{n \choose 2}x^{n-2}h^2+...+h^n\right)-x^n}{h}$$
Cancelling the $x^n$ terms and dividing out $h$ yields:
$$f'(x)=\lim_{h \to 0} \left({n \choose 1}x^{n-1}+{n \choose 2}x^{n-2}h+...+h^{n-1}\right)$$
Finally, we can consider the limit and substitute 0 in for $h$:
$$f'(x)={n \choose 1}x^{n-1}=nx^{n-1}$$
\noindent 
It's okay to not fully understand the proofs in this section; the most important thing is understanding and applying the properties themselves.
\begin{tcolorbox}
[colback=purple!5!white,colframe=purple!75!black]
\textbf{Problem 1.1} Using the power rule, show that the derivative of a constant $c$ is 0.
\end{tcolorbox}
\noindent
\textit{Solution to Problem 1.1:} We can begin by expressing the constant $c$ as $cx^0$. By the power and constant property, the derivative of this is:
\begin{align*}
f'(cx^0) &=c(x^0)' \\
         &=c(0x^{-1}) \\
         &=0
\end{align*}
\begin{tcolorbox}
[colback=purple!5!white,colframe=purple!75!black]
\textbf{Problem 1.2} Differentiate the following: \\
\noindent\\
(a)\;\;\;\;$x^5-2x^3-5x^2+12$\\
\noindent\\
(b)\;\;\;\;$x^{19}-10x^{12}+120$\\
\noindent\\
(c)\;\;\;\;$\frac{10}{x^8}+\frac{4}{x^5}-\frac{5}{x^2}$
\end{tcolorbox}
\noindent
\textit{Solution to Problem 1.2:}\\
\noindent\\
(a) By the sum property, we can begin by expressing the derivative as the sum of the derivatives of each term:
\begin{align*}
    f'(x) & = (x^5-2x^3-5x^2+12)'\\
& = (x^5)'+(-2x^3)'+(-5x^2)'+(12)'
\end{align*}
By the constant property, we can take out the constant in each term:
$$f'(x)=(x^5)'-2(x^3)'-5(x^2)'+(12)'$$
Finally, we use the power rule: 
\begin{align*}
f'(x)& = (5x^4)-2(3x^2)-5(2x)+0\\
     & = 5x^4-6x^2-10x
\end{align*}
\noindent
(b) \begin{align*}
    f'(x) &= 4(19x^{18})-10(12x^{11})+0\\
          &= 76x^{18}-120x^{11}
          \end{align*}
          \noindent
(c) We can express the function as:
\begin{align*}
    f(x) & = \frac{10}{x^8}+\frac{4}{x^5}-\frac{5}{x^2}\\
         & = 10x^{-8}+4x^{-5}-5x^{-2}
\end{align*}
From here, we proceed the same way:
$$f'(x)=-80x^{-9}-20x^{-6}+10x^{-3}$$
Note to \textit{decrease} the exponent when dealing with negative exponents.
\section{Product and quotient properties}
The product rule lets us evaluate the derivative of products. It states:\\
\noindent\\
Let $f$ and $g$ be differentiable functions. Then, $(fg)'=f'g+fg'$.
\\
\noindent\\
The quotient rule lets us evaluate the derivative of quotients of functions. It states:\\
\noindent\\
Let $f$ and $g$ be differentiable functions. Then,  $\left(\frac{f}{g}\right)'=\frac{f'g-fg'}{g^2}$.\\
\noindent\\
To prove these properties, we use the definition of derivative as well as algebraic manipulation. Although not shown here, feel free to prove these two properties yourself. \\
\noindent\\
\begin{tcolorbox}
[colback=purple!5!white,colframe=purple!75!black]
\textbf{Problem 2.1} Differentiate the following:\\
\noindent\\
(a) \;\;\;\;$(4x^3-5)(x^3-x)$\\
\noindent\\
(b) \;\;\;\;$\frac{5x^3}{x-1}$\\
\noindent\\
(c) \;\;\;\;$\frac{3x^2+x^5}{x^2+1}$
\end{tcolorbox}
\noindent
\textit{Solution to Problem 2.1:}\\
\noindent\\
(a) Using the product rule:
\begin{align*}
    f'(x) & = (4x^3-5)'(x^3-x)+(4x^3-5)(x^3-x)'\\
          & = (12x^2)(x^3-x)+(4x^3-5)(3x^2-1)\\
          & = 24x^5-16x^3-15x^2+5
\end{align*}
(b) Using the quotient rule:
\begin{align*}
    f'(x) & = \frac{(5x^3)'(x-1)-(5x^3)(x-1)'}{(x-1)^2}\\
          & = \frac{(15x^2)(x-1)-(5x^3)(1)}{(x-1)^2}\\
          & = \frac{5(2x^3-3x^2)}{(x-1)^2}
\end{align*}
(c) Using the quotient rule:
\begin{align*}
    f'(x) & = \frac{(3x^2+x^5)'(x^2+1)-(3x^2+x^5)(x^2+1)'}{(x^2+1)^2}\\
          & = \frac{(6x+5x^4)(x^2+1)-(3x^2+x^5)(2x)}{(x^2+1)^2}\\
          & = \frac{3x^6+5x^4+6x}{(x^2+1)^2}
\end{align*}
\begin{tcolorbox}
[colback=purple!5!white,colframe=purple!75!black]
\textbf{Problem 2.2} Find the equation of the line tangent to $f(x)=(1+\sqrt x)(x^2)$ at $x=4$.
\end{tcolorbox}
\noindent
\textit{Solution to Problem 2.2:} We begin by finding the corresponding $y$ value at $x=4$:
\begin{align*}
    f(4) & = (1+ \sqrt 4)(4^2)\\
         & = 48 
\end{align*}
The slope of the tangent line is the derivative of the function at $x=4$. We can find this using the product property:
\begin{align*}
    f'(x) & = (1+x^{\frac{1}{2}})'(x^2)+(1+x^{\frac{1}{2}})(x^2)'\\
          & = (\frac{1}{2}x^{-\frac{1}{2}})(x^2)+(1+x^{\frac{1}{2}})(2x)
\end{align*}
\noindent 
so
\begin{align*}
    f'(4) & = (\frac{1}{2}(4)^{-\frac{1}{2}})(4^2)+(1+4^\frac{1}{2})(2(4))\\
          & = 28
\end{align*}
Our equation is of the line with slope 28 and which passes through the point $(4,48)$:
$$y=28x-64$$
\section{Derivatives of trigonometric functions}
\noindent
We can use what we know about derivatives to expand to trigonometric functions. We will take a look at the derivative of $\sin x$:
\begin{align*}
    \frac{d}{dx}\sin x & = \lim_{h \to 0} \frac{\sin (x+h)-\sin x}{h}
\end{align*}
$\sin (x+h)$ can be expanded with the sine angle-addition identity:
\begin{align*}
    & = \lim_{h \to 0} \frac{\sin x \cos h+\sin h \cos x-\sin x}{h}\\
    & = \lim_{h \to 0} \frac{\sin x(\cos h-1)+\sin h \cos x}{h}\\
    & = \lim_{h \to 0} \frac{\sin x(\cos h-1)}{h}+\lim_{h \to 0} \frac{\sin h \cos x}{h}
\end{align*}
Here, $\sin x$ and $\cos x$ do not change based on the limit. Because of this, they are constants to the limit and can be taken out:
\begin{align*}
    & = \sin x \left(\lim_{h \to 0} \frac{\cos h-1}{h}\right)+\cos x \left(\lim_{h \to 0} \frac{\sin h}{h}\right)
\end{align*}
Previously, we looked at and solved problems with the limit
$$\lim_{x \to 0} \frac{\sin x}{x}=1$$
The proof for that limit can also be used to prove that 
$$\lim_{x \to 0} \frac{\cos x-1}{x}=0$$
Substituting those values leaves us with
$$\frac{d}{dx}\sin x=\cos x$$
The same process can be used with cosine to prove that 
$$\frac{d}{dx}\cos x=-\sin x$$
\begin{tcolorbox}
[colback=purple!5!white,colframe=purple!75!black]
\textbf{Problem 3.1} Differentiate the following:\\
\noindent\\
(a) \;\;\;\;$(2x^2+3)\cos x$ \\
\noindent\\
(b) \;\;\;\;$3x^2+x \sin x$
\end{tcolorbox}
\noindent
\textit{Solution to Problem 3.1:}\\
\noindent\\
(a) Using the product property: 
\begin{align*}
    f'(x) & = (2x^3+3)'\cos x+(2x^3+3)(\cos x)'\\
          & = 6x^2\cos x-\sin x(2x^3+3)
\end{align*}
(b) Using the sum and product property:
\begin{align*}
    f'(x) & = (3x^2)'+(x\sin x)'\\
          & = 6x+(x)'(\sin x)+(x)(\sin x)'\\
          & = 6x+ \sin x +x\cos x
\end{align*}
\begin{tcolorbox}
[colback=purple!5!white,colframe=purple!75!black]
\textbf{Problem 3.2} Evaluate $\frac{d}{dx} \tan x$. (Express tan in terms of sin and cos)
\end{tcolorbox}
\noindent
\textit{Solution to Problem 3.2:} $\tan x=\frac{\sin x}{\cos x}$. We can use the quotient rule to differentiate this:
\begin{align*}
    \frac{d}{dx} \tan x & = \frac{d}{dx}\left(\frac{\sin x}{\cos x}\right)\\
                        & = \frac{(\sin x)'(\cos x)-(\sin x)(\cos x)'}{\cos^2x}\\
                        & = \frac{\cos^2x+\sin^2x}{\cos^2x}\\
                        & = \frac{1}{\cos^2x}\\
                        & = \sec^2x
\end{align*}
\section{Derivatives of $e^x$ and $\ln x$} 
\noindent 
Listed below are two other crucial derivatives we will commonly use: 
$$\frac{d}{dx}e^x=e^x$$
$$\frac{d}{dx}\ln x=\frac{1}{x}$$
\noindent 
\begin{tcolorbox}
[colback=purple!5!white,colframe=purple!75!black]
\textbf{Problem 4.1}  Differentiate the following:\\
\noindent\\
(a) \;\;\;\;$e^x\sin x$\\
\noindent\\
(b) \;\;\;\;$3\ln x + 3x^2e^x$
\end{tcolorbox}
\noindent 
\textit{Solution to problem 4.1:} \\
\\
\noindent 
(a) Using the product property: 
\begin{align*}
    f'(x) &=(e^x)'(\sin x)+(e^x)(\sin x)' \\
          &=e^x \sin x +e^x \cos x
\end{align*} 
\noindent 
(b) Using the sum and product property: 
\begin{align*}
    f'(x) &= (3 \ln x)' + (3x^2e^x)' \\
          &=3\left(\frac{1}{x}\right)+(3x^2)'(e^x)+(3x^2)(e^x)' \\
          &=\frac{3}{x}+6xe^x+3x^2e^x
\end{align*}
\section{Recap points}
\begin{itemize}
    \item Let $f$ and $g$ be differentiable functions. Then, $(f+g)'=f'+g'$. This is the sum property. 
    \item Let $f$ be a differentiable function, and $c$ a real number. Then,  $(cf)'=c(f')$. This is the constant property. 
    \item Let $n$ be any real number. If $f(x)=x^n$, then $f'(x)=nx^{n-1}$. This is the power rule. 
    \item Let $f$ and $g$ be differentiable functions. Then, $(fg)'=f'g+fg'$. This is the product property. 
    \item Let $f$ and $g$ be differentiable functions. Then,  $\left(\frac{f}{g}\right)'=\frac{f'g-fg'}{g^2}$. This is the quotient property. 
    \item Listed below are some common derivatives: 
    $$\frac{d}{dx}\sin x=\cos x$$
    $$\frac{d}{dx}\cos x=-\sin x$$
    $$\frac{d}{dx}\tan x=\sec ^2 x$$
    $$\frac{d}{dx}e^x=e^x$$
    $$\frac{d}{dx} \ln x=\frac{1}{x}$$
\end{itemize}
\section{Exercises}\\
\noindent
\textbf{6.1} Differentiate the following:\\
\noindent\\
(a) $10x^4-12x^2+2x$\\
\noindent\\
(b) $\frac{5}{x^3}-\frac{1}{x^2}$\\
\noindent\\
(c) $(3x^2+6)(6x)$\\
\noindent\\
(d) $\frac{x^3+2}{2x}$\\
\noindent\\
(e) $2\cos x$\\
\noindent\\
(f) $2\sin x-5\cos x+7\tan x$\\
\noindent\\
(g) $\cot x$\\
\noindent\\
(h) $\sec x$\\
\noindent\\
(i) $e^x \cos x$ \\
\\
\noindent 
(j) $5x^2e^x+x\ln x$\\
\\
\noindent 
\textbf{6.2} Find the equation of the line tangent to $f(x)=\frac{2x^2+5}{5}$ at $x=5$.



\end{document}
