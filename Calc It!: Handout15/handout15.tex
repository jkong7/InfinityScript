\title{Rectilinear Motion}
\author{Jonathan Kong}
\date{}
\documentclass[11pt]{scrartcl}
\usepackage{subfiles}
\usepackage[sexy]{evan}
\usepackage[utf8]{inputenc}
\usepackage{ upgreek }
\usepackage[english]{babel}
\usepackage{ragged2e}
\usepackage{blindtext}
\usepackage{geometry}
\geometry{%}
  letterpaper,
  lmargin=1.5cm,
  rmargin=1.5cm,
  tmargin=2 cm,
  bmargin=2cm,
  footskip=12pt,
  headheight=13.6pt}
\usepackage{url}
\urlstyle{tt}
\usepackage{float}
\usepackage{verbatim}
\usepackage[english]{babel}
\usepackage[margin=1in]{geometry}
\usepackage{amsmath}
\usepackage{tcolorbox}
\usepackage[dvipsnames]{xcolor}
\usepackage{amssymb}\usepackage{dcolumn}
\newcolumntype{2}{D{.}{}{2.0}}
\begin{document}
\maketitle
\noindent

\section{Defining position, velocity, and acceleration}
\noindent
In this section, we will use calculus to analyze rectilinear motion, which is defined as the motion of a particle along a linear path. This path can be the $x$/$y$-axis or it can be an inclined line. \\
\\
\noindent 
We can define $s$ to be the \textbf{position function} of the particle, and subsequently, $s(t)$ tells us the location of the particle at some time $t$. Next, we define $v$ to be \textbf{velocity} function of the particle. Velocity is defined as the change in position. For example, if $s(2)=5$ and $s(4)=10$, the \textbf{average velocity} over the time period $t=2$ to $t=4$ is 
$$\frac{s(4)-s(2)}{4-2}=\frac{5}{2}$$
To determine the \textbf{instantaneous} velocity at some time $t$, velocity becomes the derivative of the position function: 
$$v(t)=s'(t)$$
Lastly, we can define $a$ to be the \textbf{acceleration function} of the particle, and this is defined as the rate at which the velocity changes with time. Therefore, the instantaneous acceleration of a particle at some time $t$ is defined as the derivative of the instantaneous velocity: 
$$a(t)=v'(t)=s''(t)$$
\begin{tcolorbox}
[colback=purple!5!white,colframe=purple!75!black]
\textbf{Problem 1.1} The position of a particle is described by $s(t)=5t^4-3t^2+3t$. Find the velocity and acceleration functions. 
\end{tcolorbox}
\noindent 
We are given the position function and we know its relation to the velocity and acceleration functions are given through differentiation: 
$$v(t)=s'(t)=20t^3-6t+3$$
$$a(t)=v'(t)=60t^2-6$$
We now continue with a few more crucial rectilinear motion concepts. 
\section{Defining speed, displacement, and distance}
\textbf{Speed} is defined as the magnitude of velocity. Speed is a scalar quantity, which means that we are not concerned with direction or sign as opposed to velocity: 
$$\text{Speed}=\lvert v(t) \rvert = \lvert  s'(t) \rvert $$
\textbf{Displacement} is defined as the change in position of a particle over a time period. The displacement of a particle over a time interval $[t_1,t_2]$ can be found through the following expression:
$$s(t_2)-s(t_1)$$
Where as displacement is simply how far the particle is from its original position, \textbf{distance} takes into account the entire path travelled by the particle. \\
\\
\noindent 
For example, if a particle starts from the origin and travels 4 units north, and then 3 units south, and finally 6 units north, its displacement is the difference between its final and initial position: 
$$7-0=7$$
The distance it travelled takes into account the entire travlelled path: 
$$4+3+6=13$$
\section{Rest, direction change, and speeding up/slowing down}
\noindent 
We say that an object is at \textbf{rest} when its velocity is 0. An object \textbf{changes direction} when the sign of its velocity changes. An object is \textbf{speeding up} when the signs of its acceleration and velocity match. An object is \textbf{slowing down} when the signs don't match. \\
\\
\noindent 
Try to make sense of this. Speed is independent of direction, so even if the signs of velocity and acceleration are both negative, the magnitude of the particle's velocity is still increasing, and so speed is positive. We now will look at a problem that ties in everything we have covered. 
\begin{tcolorbox}
[colback=purple!5!white,colframe=purple!75!black]
\textbf{Problem 3.1} The position of a particle is given by $s(t)=t^3-9t^2+15t$ for $0 \leq t \leq 12$. \\
\noindent 
(a) Find the velocity and acceleration functions. \\
\noindent 
(b) At what times is the particle at rest? \\
\noindent 
(c) At what times do the particle change direction? \\
\noindent 
(d) At what times is the particle speeding up and at what times is it slowing down? \\
\noindent 
(e) Find the displacement of the particle over the indicated time interval. \\
\noindent 
(f) Find the total distance travelled by the particle over the indicated time interval. 
\end{tcolorbox}
\noindent
\textit{Solution to problem 3.1:} \\
\\
\noindent 
(a) We note that velocity is the derivative of position and that acceleration is the derivative of velocity to solve: 
$$v(t)=s'(t)=3t^2-18t+15$$
$$a(t)=v'(t)=6t-18$$\\
\noindent 
(b) The particle is at rest when velocity is 0. Therefore, we set our velocity function to be 0 and solve: 
$$v(t)=0=3t^2-18t+15$$
$$t^2-6t+5=0$$
$$(t-5)(t-1)=0$$
$$t=5, 1$$
The particle is at rest at times $t=1$ and $t=5$.\\
\\
\noindent 
(c) The particle changes direction when the sign of velocity changes. We therefore first find the roots of the velocity function:
$$v(t)=0=3t^2-18t+15$$
$$t=5,1$$
\noindent 
Next, to test whether the sign of velocity changes at these points, we create a sign chart: \\
\newdimen\tcolw \tcolw=2.5em % the column width
\edef\ecatcode{\catcode`&=\the\catcode`&\relax}\catcode`&=4
\def\sgchart#1#2{\vbox{\offinterlineskip\halign{\hfil##\quad&##\hfil\crcr\sgchartA#2,:,%
   \omit\sgchartR&\kern.2pt\sgchartS{.5\tcolw}\relax\sgchartE#1,\relax,%
   \sgchartS{.5\tcolw}\relax\cr
   \noalign{\kern2pt}&\def~{}\kern.5\tcolw\sgchartD#1,\relax,\cr}}}
\def\sgchartA#1:#2,{\cr\ifx,#1,\else $#1$&\sgchartB#2{}\expandafter\sgchartA\fi}
\def\sgchartB#1{\hbox to\tcolw{\hss$#1$\hss}\sgchartC}
\def\sgchartC#1{\ifx,#1,\else
   \strut\vrule\kern-.4pt\hbox to\tcolw{\hss$#1$\hss}\expandafter\sgchartC\fi}
\def\sgchartD#1#2,{\ifx\relax#1\else\hbox to\tcolw{\hss$#1#2$\hss}\expandafter\sgchartD\fi}
\def\sgchartE#1#2,{\ifx\relax#1\else
    \ifx~#1\sgchartS\tcolw\circ \else\sgchartS\tcolw\bullet\fi \expandafter\sgchartE\fi}
\def\sgchartR{\leaders\vrule height2.8pt depth-2.4pt\hfil}
\def\sgchartS#1#2{\hbox to#1{\kern-.2pt\sgchartR \ifx\relax#2\else
   \kern-.7pt$#2$\kern-.7pt\sgchartR\fi\kern-.2pt}}
\ecatcode
\begin{center}
\sgchart{1,5}  {v(t): +-+}
\end{center}
\noindent
Because the sign of $v(t)$ changes at both $t=1$ and $t=5$, the particle changes direction at those times. \\
\\
\noindent 
(d) To solve this, we need to compare the signs of velocity and acceleration. Because we have already made a sign chart for velocity, we only need to analyze acceleration. We begin by finding the roots of the acceleration function: 
$$a(t)=0=6t-18$$
$$t=3$$
We now must compare side-by-side the signs of velocity and acceleration across the entire interval broken up by potential sign changes of either function. 
\noindent 
We can achieve this through a combined sign chart: \\ 
\newdimen\tcolw \tcolw=2.5em % the column width
\edef\ecatcode{\catcode`&=\the\catcode`&\relax}\catcode`&=4
\def\sgchart#1#2{\vbox{\offinterlineskip\halign{\hfil##\quad&##\hfil\crcr\sgchartA#2,:,%
   \omit\sgchartR&\kern.2pt\sgchartS{.5\tcolw}\relax\sgchartE#1,\relax,%
   \sgchartS{.5\tcolw}\relax\cr
   \noalign{\kern2pt}&\def~{}\kern.5\tcolw\sgchartD#1,\relax,\cr}}}
\def\sgchartA#1:#2,{\cr\ifx,#1,\else $#1$&\sgchartB#2{}\expandafter\sgchartA\fi}
\def\sgchartB#1{\hbox to\tcolw{\hss$#1$\hss}\sgchartC}
\def\sgchartC#1{\ifx,#1,\else
   \strut\vrule\kern-.4pt\hbox to\tcolw{\hss$#1$\hss}\expandafter\sgchartC\fi}
\def\sgchartD#1#2,{\ifx\relax#1\else\hbox to\tcolw{\hss$#1#2$\hss}\expandafter\sgchartD\fi}
\def\sgchartE#1#2,{\ifx\relax#1\else
    \ifx~#1\sgchartS\tcolw\circ \else\sgchartS\tcolw\bullet\fi \expandafter\sgchartE\fi}
\def\sgchartR{\leaders\vrule height2.8pt depth-2.4pt\hfil}
\def\sgchartS#1#2{\hbox to#1{\kern-.2pt\sgchartR \ifx\relax#2\else
   \kern-.7pt$#2$\kern-.7pt\sgchartR\fi\kern-.2pt}}
\ecatcode
\begin{center}
\sgchart{1,3,5}  {a(t): --++, v(t): +--+}
\end{center}
The particle is slowing down when the signs of velocity and acceleration don't match and speeding up when the signs match. The function is therefore slowing down on the intervals $0 < t <1$ and $3 < t <5$ and speeding up on the intervals $1 < t <3$ and $5 < t < 12$. \\
\\
\noindent 
(e) To find displacement, we simply find the difference between the final and initial position over the time interval: 
$$\text{Displacement}=s(12)-s(0)=612-0=612$$
\noindent 
(f) To calculate distance, we cannot simply subtract the final and initial position. This is because the changes in direction are ignored. Instead, we sum all segments of travel based on the times where the particle changes direction: 
\begin{align*}
    \text{Distance}=\lvert s(12)-s(5)\rvert + \lvert s(5)-s(1) \rvert + \lvert s(1)-s(0) \rvert &=\lvert 612-(-25) \rvert + \lvert -25+7 \rvert + \lvert 7-0 \rvert \\
           &=662
\end{align*}
Note that we must use absolute value as direction is not a factor when calculating distance. 
\section{Recap points}
\begin{itemize}
    \item Calculus can be used to analyze rectilinear motion, which is the motion of a particle along a linear path. 
    \item Position, velocity, and acceleration are commonly used to describe a particle's motion. They are related by differentiation: 
    $$a(t)=v'(t)=s''(t)$$
    \item Speed is the magnitude of velocity. This means that direction is not a factor when calculating speed. Displacement is the change in position and is found by subtracting the initial position from the final position. Distance is the sum of the segments constituting the entire travelled path from one position to another. 
    \item An object is at rest when its velocity is 0, changes direction when the sign of its velocity changes, speeds up when the signs of its acceleration and velocity match, and slows down when the signs of its acceleration and velocity differ. 
    \item Rectilinear motion problems make use of differentiation, finding zeroes of functions, sign charts, and absolute value signs. 
\end{itemize}
\section{Exercises} \\
\noindent 
\textbf{5.1} The position of a particle is given by $s(t)=t^3-3t^2$ for $0\leq t \leq10$\\
\\
\noindent 
(a) Find the velocity and acceleration functions. \\
\\
\noindent 
(b) At what times is the particle at rest? \\
\\
\noindent 
(c) At what times do the particle change direction? \\
\\
\noindent 
(d) At what times is the particle speeding up and at what times it slowing down? \\
\\
\noindent 
(e) Find the displacement of the particle over the indicated time interval. \\
\\
\noindent 
(f) Find the total distance travelled by the particle over the indicated time interval.
\end{document}
