\title{Introduction to Limits}
\author{Jonathan Kong}
\date{}
\documentclass[11pt]{scrartcl}
\usepackage{subfiles}
\usepackage[sexy]{evan}
\usepackage[utf8]{inputenc}
\usepackage{ upgreek }
\usepackage{geometry}
\geometry{%}
  letterpaper,
  lmargin=1.5cm,
  rmargin=1.5cm,
  tmargin=2 cm,
  bmargin=2cm,
  footskip=12pt,
  headheight=13.6pt}
\usepackage{url}
\urlstyle{tt}
\usepackage{float}
\usepackage{verbatim}
\usepackage[margin=1in]{geometry}
\usepackage{amsmath}
\usepackage{tcolorbox}
\usepackage[dvipsnames]{xcolor}
\usepackage{amssymb}\usepackage{dcolumn}
\newcolumntype{2}{D{.}{}{2.0}}
\begin{document}
\maketitle
\noindent

\section{What is a limit?}
\noindent
The concept of the limit is generally the first topic covered in a calculus course. It is essential for later topics in derivative and integrals. A limit describes the behavior of a function as it \textit{approaches} a certain value. We can visualize this by looking at a simple example. 
\begin{figure}[htp]
    \centering
    \includegraphics[width=12cm]{yuh.png}
\end{figure} \\
\\
\noindent 
The graph above shows that as $x$ approaches 2, the value of the function approaches 4. In other words, for $x$ values very close to 2, the value of the function is very close to 4. In limit notation, this is written as 
$$\lim_{x \to 0} f(x)=1$$ 
\noindent 
The graph of a function can help a lot in visualizing limits. It helps us to see the behavior of a function at certain areas which ultimately can help us find limits. \\


\begin{tcolorbox}
[colback=purple!5!white,colframe=purple!75!black]
\textbf{Problem 1.1} Consider the function $f(x)=2x^2$. Evaluate $\lim_{x \to 2} f(x)$.
\end{tcolorbox}
\noindent
\textit{Solution to Problem 1.1:} The graph of $f(x)=2x^2$ is a parabola with vertex at the origin. As $x$ approaches 2, the function approaches the value $f(2)=2(2)^2=8$. In other words, for $x$ values very close to 2, the function is equal to values very close to 8. \\

\begin{tcolorbox}
[colback=purple!5!white,colframe=purple!75!black]
\textbf{Problem 1.2} Consider the function $f(x)=\frac{x^2-9}{x-3}$. Evaluate $\lim_{x \to 3} f(x)$.
\end{tcolorbox}
\noindent
\textit{Solution to Problem 1.2:} We can begin by plugging in 3 to the function. However, this gives us $f(3)=\frac{0}{0}$, an indeterminate value. Factoring the numerator, we see that we can remove $(x-3)$ from the numerator and denominator. This means that the graph of our function is the line $y=x+3$ with a hole at the point (3,6). As $x$ approaches 3, the function is still approaching 6 despite the hole. The limit is then 6. \\
\\
\noindent 
This leads us to our first discovery about limits: The value of $\lim_{x \to a} f(x)$ is not concerned by what $x$ is at $a$, but what $x$ is near $a$. In the limit above, the value of the function at $x=3$ could have been anything and the limit would still be the same. 
\\
\begin{tcolorbox}
[colback=purple!5!white,colframe=purple!75!black]
\textbf{Problem 1.3} Consider the function
\[
   f(x)= \begin{dcases}
        x & x\leq 0 \\
        x+1 & x > 0
       
    \end{dcases}
\]
Determine the value of $\lim_{x \to 0} f(x)$
\end{tcolorbox}
\noindent
\textit{Solution to Problem 1.3:} Looking at the graph of this function, the left side of $x=0$ is the graph $y=x$ and the right side is $y=x+1$. From the left side, the limit as x approaches 0 is 0. However, from the right side, it approaches 1. The difference in values of the one-sided limits leads to the fact that the limit does not exist at $x=0$.\\
\noindent\\
\\$f(x)$ has limit $L$ when $x$ approaches $a$ from the left:
\\$$\lim\limits_{x \to a^{-}} f(x)=L$$
\\$f(x)$ has limit $L$ when $x$ approaches $a$ from the right:
\\$$\lim\limits_{x \to a^{+}} f(x)=L$$
\\The limit exists if and only if the one-sided limits are equal:
\\$$\lim\limits_{x \to a^{\vphantom{+}}} f(x)
\Longleftrightarrow \lim\limits_{x \to a^{-}} f(x)
= \lim\limits_{x \to a^{+}} f(x)$$
\noindent
Below are some of the basic properties of limits:\\
$$\lim_{x \to a } \left[k \cdot f(x)\right]=k \cdot\lim_{x \to a} f(x)$$
$$\lim_{x \to a} \left[(f(x)\pm g(x)\right]=\lim_{x \to a} f(x)\pm \lim_{x \to a} g(x)$$
$$\lim_{x \to a} \left[(f(x)\cdot g(x)\right]=\lim_{x \to a} f(x)\cdot \lim_{x \to a} g(x)$$
$$\lim_{x \to a} \left[(f(x)/g(x)\right]=\lim_{x \to a} f(x)/\lim_{x \to a} g(x),\;\;\;\; \lim_{x \to a} g(x)\neq0$$

\section{Evaluating limits}
\noindent
In this section, we will take a look at some strategies to evaluate various types of limits. 
\begin{tcolorbox}
[colback=purple!5!white,colframe=purple!75!black]
\textbf{Problem 2.1} Evaluate the following limits:\\

(a)   $\lim_{x \to 3} \frac{x^2-2x+5}{4}$\\

(b)   $\lim_{x \to 2} \frac{(x+2)^2-1}{3}$
\end{tcolorbox}
\noindent
\textit{Solution to Problem 2.1:}\\
\noindent
\\
(a)   Plugging 3 into the function, we obtain $\lim_{x \to 3} \frac{x^2-2x+5}{4}=\frac{3^2-2(3)+5}{4}=2$
\\
\\
\noindent 
(b)   Plugging 2 into the function, we obtain $\lim_{x \to 2} \frac{(x+2)^2-1}{3}=\frac{4^2-1}{3}=5$\\
\\
\noindent 
Next, we take a look at some limits for which direct substitution fails:
\\
\begin{tcolorbox}
[colback=purple!5!white,colframe=purple!75!black]
\textbf{Problem 2.2} Evaluate the following limits:\\

(a) $\lim_{x \to 4} \frac{x^2-2x-8}{x-4}$\\

(b) $\lim_{x \to 0} \frac{\frac{1}{x+2}-\frac{1}{2}}{x}$
              
\end{tcolorbox}
\noindent
\textit{Solution to Problem 2.2}\\
\noindent
\\
(a)   Plugging four into the function gives $\frac{0}{0}$ which has \textit{indeterminate form}. We can factor $x-4$
out of the function and then evaluate the limit:
$$\frac{x^2-2x-8}{x-4}=\frac{(x+2)(x-4)}{x-4}=x+2$$
\noindent
Now that we have the function in this simplified form, we can perform direct substitution:
$$\lim_{x \to 4} \frac{x^2-2x-8}{x-4}= \lim_{x \to 4} (x+2)=6$$ 
\noindent
(b) We begin by noting that direct substitution leaves us with an indeterminate value. We see that there are two fractions in the numerator of the function so we can begin by combining them:
$$\frac{\frac{1}{x+2}-\frac{1}{2}}{x}=\frac{\frac{2-(x+2)}{2(x+2)}}{x}=-\frac{1}{2x+4}$$
\noindent
By combining the fractions in the numerator, the function is in a form that we can now perform direct substitution:
$$\lim_{x \to 0} \frac{\frac{1}{x+2}-\frac{1}{2}}{x}=\lim_{x \to 0} -\frac{1}{2x+4}=-\frac{1}{4}$$
\noindent
\\
\begin{tcolorbox}
[colback=purple!5!white,colframe=purple!75!black]
\textbf{Problem 2.3} Evaluate the following limits:\\

(a) $\lim_{x \to 1} \frac{1-\sqrt[]{x}}{1-x}$\\

(b) $\lim_{x \to 3} \frac{\abs{x-3}}{x-3}$
\end{tcolorbox}
\noindent
\textit{Solution to Problem 2.3:}\\
\noindent
\\
(a) Since plugging 1 into the function yields an indeterminate value, we must give it another form. The function stands out in that it has a square root. We can get rid of it in the numerator by multiplying the function by its conjugate: 
$$\frac{(1-\sqrt[]{x})(1+\sqrt[]{x})}{(1-x)(1+\sqrt[]{x})}=\frac{1-x}{(1-x)(1+\sqrt x)}=\frac{1}{1+\sqrt x}$$
\noindent
From here, we direct substitute:
$$\lim_{x \to 1} \frac{1-\sqrt[]{x}}{1-x}=\lim_{x \to 1} \frac{1}{1+\sqrt[]{x}}=\frac{1}{2}$$
\noindent
(b) Since plugging 3 into the function yields an indeterminate value, we must give it another form. It appears there is no algebraic method of simplifying the function so we must interpret the absolute value. We can split the limit into one-sided limits at $x=3$:
$$\lim\limits_{x \to 3^{-}} \frac{\abs{x-3}}{x-3} \Longleftrightarrow \lim\limits_{x \to 3^{+}} \frac{\abs{x-3}}{x-3}$$
Lets first examine the left limit. The numerator of the function always stays positive due to the absolute value. However, the limit approaching 3 from the left means that $x-3$ is negative since $x<3$. Because the magnitude of the numerator and denominator are the same and their signs are opposite, the left limit is -1.\\
\noindent
\\
In the right limit, the denominator is positive since $x>3$.
The magnitude of the numerator and denominator are the same and their signs are the same. Because of this, the right limit is 1.\\
\noindent
\\
Since the one-sided limits have different values, the limit does not exist. \\
\\
\noindent 
The next limit will deal with x approaching infinity. Computation with infinity brings some new indeterminate forms such as $\infty-\infty$ and $\frac{\infty}{\infty}$. With this in mind, we can solve the next problem:
\\
\begin{tcolorbox}
[colback=purple!5!white,colframe=purple!75!black]
\textbf{Problem 2.4} Evaluate $\lim_{x \to \infty} (3\:\sqrt[]{x}-x)$
\end{tcolorbox}
\noindent
\textit{Solution to Problem 2.4:}
Direct substituting, we take a look at the terms $3\:\sqrt[]{\infty}$ and -\infty. \;$3\:\sqrt[]{\infty}$ can be written as just $\infty$ since it grows infinitely large. We are then left with $\infty-\infty$, an indeterminate form. Even if the two terms are both infinitely large, we might notice that $3\:\sqrt[]{\infty}$ is less than $\infty$ so our answer should be -$\infty$. We then look for ways to find this answer. The only way to algebraically manipulate the function is to factor $\sqrt[]{x}$:
$$3\:\sqrt[]{x}-x=\sqrt[]{x}(3-\sqrt[]{x})$$
$$\lim_{x \to \infty} (3\:\sqrt[]{x}-x)=\lim_{x \to \infty} \sqrt[]{x}(3-\sqrt[]{x})=(\infty)(-\infty)=-\infty$$
\begin{tcolorbox}
[colback=purple!5!white,colframe=purple!75!black]
\textbf{Problem 2.5} Evaluate $\lim_{x \to \infty} \frac{7^x-5^x}{8^x}$
\end{tcolorbox}
\noindent
\textit{Solution to Problem 2.5:} Direct substituting, we find that the numerator and the denominator are both $\infty$ which leads to the indeterminate value $\frac{\infty}{\infty}$. We can begin by splitting the function into two fractions:
$$\frac{7^x-5^x}{8^x}=\frac{7^x}{8^x}-\frac{5^x}{8^x}=\left(\frac{7}{8}\right)^x-\left(\frac{5}{8}\right)^x$$
A fraction raised to an ever growing value approaches 0. Because of this, we can direct substitute to obtain:
$$\lim_{x \to \infty} \frac{7^x-5^x}{8^x}=\lim_{x \to \infty} \left[\left(\frac{7}{8}\right)^x-\left(\frac{5}{8}\right)^x\right]=0$$
\noindent
   The next few limits will be based on this very important limit:
   $$\lim_{x \to 0} \frac{\sin x}{x}=1$$
\noindent
This limit can be proved using a theorem known as the squeeze theorem. 
\\
\begin{tcolorbox}
[colback=purple!5!white,colframe=purple!75!black]
\textbf{Problem 2.6} Evaluate the following limits:\\

(a) $\lim_{x \to 0} \frac{\sin 4x}{9x}$\\

(b) $\lim_{x \to 0} \frac{\sin 3x^2}{x}$
\end{tcolorbox}
\noindent
\textit{Solution to Problem 2.6:}\\
\noindent\\
(a) Direct substitution yields $\frac{0}{0}$. We want to make the function take the form of $\frac{\sin y}{y}$ since we know the limit of this is 1. We can do this by replacing the 9 in the denominator for 4 and putting it outside. In this case, $y=4x$:
$$\frac{\sin 4x}{9x}=\frac{\sin 4x}{4x}\cdot\frac{4}{9}$$
Now that we have our desired form, we can continue:
$$\lim_{x \to 0} \frac{\sin 4x}{9x}=\frac{4}{9}\cdot\lim_{x \to 0} \frac{\sin 4x}{4x}=\frac{4}{9}$$
\noindent
(b) Just like in part a, we must make our function take the form of $\frac{\sin y}{y}$. Multiplying the denominator by $3x$ gives us this form thus we multiply the function by $\frac{3x}{3x}$:
$$\frac{\sin 3x^2}{x}=\frac{\sin 3x^2}{3x^2}\cdot3x$$
Now that we have our desired form, we can continue. Note that we are able to make the limit into two due to properties of limits:
$$\lim_{x \to 0} \frac{\sin 3x^2}{x}=\lim_{x \to 0} \frac{\sin 3x^2}{3x^2}\cdot\lim_{x \to 0} 3x=1\cdot0=0$$
\noindent\\
In this section, we took a look at many types of limits. Even though there are many more families of limits not covered here, this section should serve as a general introduction on the tools that can be used to evaluate limits.
\section{Exercises}\\
\noindent
\textbf{3.1} Evaluate the following limits:\\
\noindent\\
(a) $\lim_{x \to \infty} \frac{5x^2-15x}{4x-12}$\\
\noindent\\
(b) $\lim_{x \to \infty} \frac{6x^2-5x}{3x^2+6}$\\
\noindent\\
\textbf{3.2} Evaluate the following limits:\\
\noindent\\
(a) $\lim_{x \to 0} \frac{x}{\sqrt{x+16}-4}$\\
\noindent\\
(b) $\lim_{x \to 9} \frac{\frac{1}{\sqrt{x}}-\frac{1}{3}}{x-9}$\\
\noindent\\
\textbf{3.3} Evaluate the following limits:\\
\noindent\\
(a) $\lim_{x \to \pi} \sin (\sin x +x)$\\
\noindent\\
(b) $\lim_{x \to 2} \frac{\sin (x-2)}{x^2+2x-8}$\\
\noindent\\
\textbf{3.4} Evaluate the limit $\lim_{x \to \infty} \frac{n!}{(n+1)!-n!}$ where $n!$ refers to the product of all positive integers less than or equal to $n$. $n!=n(n-1)(n-2)(n-3)...(2)(1)$. \\
\noindent\\




\end{document}


