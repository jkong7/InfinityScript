\title{Area Between Curves}
\author{Jonathan Kong}
\date{}
\documentclass[11pt]{scrartcl}
\usepackage{subfiles}
\usepackage[sexy]{evan}
\usepackage[utf8]{inputenc}
\usepackage{ upgreek }
\usepackage{geometry}
\geometry{%}
  letterpaper,
  lmargin=1.5cm,
  rmargin=1.5cm,
  tmargin=2 cm,
  bmargin=2cm,
  footskip=12pt,
  headheight=13.6pt}
\usepackage{url}
\urlstyle{tt}
\usepackage{float}
\usepackage[margin=1in]{geometry}
\usepackage{verbatim}
\usepackage{amsmath}
\usepackage{tcolorbox}
\usepackage[dvipsnames]{xcolor}
\usepackage{amssymb}\usepackage{dcolumn}
\newcolumntype{2}{D{.}{}{2.0}}
\begin{document}
\maketitle
\noindent 

\section{Definite integrals and area between curves}
\noindent
In this section, we will find the area between curves using definite integrals. \\
\\
\noindent 
To start, take a look at the image below: 

\begin{figure}[htp]
    \centering
    \includegraphics[width=11cm]{Screenshot (480).png}
    
\end{figure}
\noindent 
The area bounded by the graph of $f$ on $[a,b]$ can be represented by the definite integral $\int_{a}^{b} f(x) \ dx$. Similarly the area bounded by the graph of $g$ on $[a,b]$ can be represented by the definite integral $\int_a^b g(x) \ dx$. When you subtract the area under $g$ by the area under $f$, you get the area between the two curves. This area is therefore 
$$\int_a^b {[f(x)-g(x)]} \ dx$$
We note that in using this expression, we always subtract the bottom function from the top. The proper result is as follows: \\
\\
\noindent 
The area of the region bounded above by $y=f(x)$, below by $y=g(x)$, on the left by $x=a$, and on the right by $x=b$, where $f$ and $g$ are continuous functions on $[a,b]$ such that $f(x)\ge g(x)$ for all x in $[a,b]$, is 
$$\int_a^b [f(x)-g(x)] \ dx$$
\noindent 
This intuitively should make a lot of sense. We are simply subtracting two areas to find the overlapping area, a concept commonly used in geometry problems. 
\section{Area between curves problems}
\noindent 
For any area between curves problem, an ideal first step is to create a sketch of the region. After this, setting up the integral expression should follow naturally and easily. You should not have to memorize the area between curves formula if you can understand and conceptualize every problem. 
\begin{tcolorbox}[colback=purple!5!white,colframe=purple!75!black]
\textbf{Problem 2.1} Find the area of the region enclosed by $y=x^2$ and $y=4x-x^2$. 
\end{tcolorbox}
\noindent 
\textit{Solution to problem 2.1:} We begin with a sketch of the region: 

\begin{figure}[htp]
    \centering
    \includegraphics[width=10cm]{Screenshot (481).png}

\end{figure}
\noindent 
Next, we need to find the limits of integration, which here are simply the intersection points between the two curves. We find these by equating the functions and solving: 
$$x^2=4x-x^2$$
$$x=0,2$$
\noindent 
Lastly, we note that $y=4x-x^2$ is the \say{upper} function in the bounded region. With this, we have all the information needed to set up and solve an integral expression:
\begin{align*}
    \text{Area} &= \int_a^b [f(x)-g(x)] \ dx \\
                &= \int_0^2 [4x-x^2-x^2] \ dx \\
                &= \left(2x^2-\frac{2}{3}x^3\right) \biggr \rvert_0^2 \\
                &=\frac{8}{3}
\end{align*}
The process we used above can be generalized as follows: 
\begin{enumerate}
    \item Create a sketch of the curves and desired region
    \item Determine the limits of integration 
    \item Determine which function you are to subtract from the other in the integration expression 
    \item Set up and solve the area integration expression 
\end{enumerate}
\noindent 
Of course, not all problems of this sort are this cut and dry. We will soon look at a few problems that require a few more steps and thought.
\section{Swapping the roles of $x$ and $y$}
\noindent 
We can also solve for the area between curves using another perspective. Instead of integrating with respect to $x$, we can integrate with respect to $y$. \\
\\
\noindent 
The proper result for finding area between curves this way is as follows: \\
\\
\noindent 
The area of the region bounded on the left by $x=v(y)$, on the right by $x=w(y)$, and above and below by $y=d$ and $y=c$, where $w$ and $v$ are continuous functions on $[c,d]$ such that $w(y) \ge v(y)$ for all $y$ in $[c,d]$, is 
$$\int_c^d[w(y)-v(y)]dy$$
\noindent 
Here, instead of being vertically bounded, the areas are horizontally bounded. Therefore, we subtract the left function from the right to find the area between two curves. The limits of integration are the $y$ values of the intersections, with the lower limit being the lower $y$-value. \\
\\
\noindent 
Why integrate with respect to $y$? Often, finding an area by integrating with respect to $x$ requires you to split the region into two because of inconsistent boundaries. Integrating with respect to $y$ eliminates this. 
\begin{tcolorbox}[colback=purple!5!white,colframe=purple!75!black]
\textbf{Problem 3.1} Find the area of the region enclosed by $x=-y$, $x=2-y^2$, and the $x$-axis. 
\end{tcolorbox}
\noindent 
\textit{Solution to problem 3.1:} We begin with a sketch of the region:\\
\\
\begin{figure}[htp]
    \centering
    \includegraphics[width=10cm]{Screenshot (490).png}
\end{figure} \\
\noindent 
We know that one of our limits of integration is $y=0$. The other is the intersection point: 
$$-y=2-y^2$$
$$(y-2)(y+1)=0$$
$$y=2,-1$$
$y=-1$ can be omitted as it's outside the area boundary. Here, $x=2-y^2$ is our \say{right} function. We now have all the information to set up and solve an integral expression: 
\begin{align*}
    \text{Area} &= \int_c^d[w(y)-v(y)] \ dy \\
                &=\int_0^2[2-y^2-(-y)] \ dy \\
                &=\left(-\frac{1}{3}y^3+\frac{1}{2}y^2+2y\right)\biggr \rvert_0^2 \\
                &=\frac{10}{3}
\end{align*}
\section{A harder problem}
\begin{tcolorbox}[colback=purple!5!white,colframe=purple!75!black]
\textbf{Problem 4.1} Find $b$ such that the line $y=b$ divides the area enclosed by $y=x^2$ and $x=9$ into two equal regions.
\end{tcolorbox}
\noindent 
\textit{Solution to problem 3.1} We begin with a sketch of the region. 

\begin{figure}[htp]
    \centering
    \includegraphics[width=12cm]{Screenshot (492).png}
\end{figure}
\\
\noindent 
Note that the line $y=b$ will be the upper boundary for the lower region and the lower boundary for the upper region. To find $b$ then, we find the area expression for each region and then equate them. \\
\\
\noindent 
Because the region is symmetric about the $y$-axis, it is only necessary to analyze one of the \say{half}-regions. For purposes of equating then, we will look only at the region to the right of the $y$-axis. \\
\\
\noindent 
Because there is a consistent \say{left} boundary from $0 \le y \le 9$, it is easier to integrate with respect to $y$ and so we use the equation $x=\sqrt{y}$. The lower region has limits of integration $0$ and $b$ so its area is given by  
$$\text{Area}=\int_0^b \sqrt{y} \ dy$$
The upper region has limits of integration $b$ and 9 so its area is given by
$$\text{Area}=\int_b^9 \sqrt{y} \ dy$$
We equate the two expressions and solve for $b$: 
$$\int_0^b \sqrt{y} \ dy=\int_b^9 \sqrt{y} \ dy$$
$$\left(\frac{2}{3}y^\frac{2}{2}\right) \biggr \rvert_0^b=\left(\frac{2}{3}y^\frac{3}{2}\right) \biggr \rvert_b^9$$
$$\frac{2}{3}b^\frac{3}{2}=\frac{2}{3}(27-b^\frac{3}{2})$$
$$b=\frac{9}{\sqrt[3]{4}}$$
\noindent 
\section{Recap points}
\begin{itemize}
    \item The area between two curves can be found by subtracting the areas bounded by each individual curve. This concept is similar to geometry problems where we subtract two areas to find one overlapping area. 
    \item When finding the area between two curves in an interval $[a,b]$ by integrating with respect to $x$, we use the following expression: 
    $$\int_a^b [f(x)-g(x)] \ dx$$
    \item When finding the area between two curves in a vertical interval $[c,d]$ by integrating with respect to $y$, we use the following expression: 
    $$\int_c^d[w(y)-v(y)]dy$$
    \item Begin all area between curves problems with a sketch of the desired region. 
    \item It is often much easier and quicker to use one of the two integration expressions depending on the boundary of the desired area. 
\end{itemize}
\section{Exercises}\\
\\
\noindent 
\textbf{6.1} Find the area of the region enclosed by $y=x^2-4x-5$ and $y=2x-5$. \\
\\
\noindent 
\textbf{6.2} Find the area of the region enclosed by $x=\sqrt{y}$, $x=y-2$, and the $x$-axis.  \\
\\
\noindent 
\textbf{6.3} Find $m$ such that the line $y=mx$ divides the area enclosed by $y=x-x^2$ and the $x$-axis into two equal regions. 
\end{document}