\title{Volumes by Cross Section}
\author{Jonathan Kong}
\date{}
\documentclass[11pt]{scrartcl}
\usepackage{subfiles}
\usepackage[sexy]{evan}
\usepackage[utf8]{inputenc}
\usepackage{ upgreek }
\usepackage{geometry}
\geometry{%}
  letterpaper,
  lmargin=1.5cm,
  rmargin=1.5cm,
  tmargin=2 cm,
  bmargin=2cm,
  footskip=12pt,
  headheight=13.6pt}
\usepackage{url}
\urlstyle{tt}
\usepackage{float}
\usepackage{verbatim}
\usepackage{amsmath}
\usepackage{tcolorbox}
\usepackage[dvipsnames]{xcolor}
\usepackage{amssymb}\usepackage{dcolumn}
\newcolumntype{2}{D{.}{}{2.0}}
\begin{document}
\maketitle
\noindent 

\section{Definite integrals and cross sections}
\noindent
In this section, we will find the volume of solids using cross sections. Just like with finding area between curves, we will be utilizing definite integrals.\\
\\
\noindent 
To start, take a look at the image below: 

\begin{figure}[htp]
    \centering
    \includegraphics[width=12cm]{Screenshot (517).png}
\end{figure} \\
\noindent 
As seen, the cross sections of the solid $S$ are different in area. We can represent the area of the cross sections as a function of $x$: $A(x)$. If each cross section strip was infinitesimally small and we \say{added} all strips from one end $a$ to another end $b$, the result would be the volume of the solid. \\
\\
\noindent The \say{adding} of the strips can be represented with a definite integral. The proper result is as follows: \\
\\
\noindent 
The volume of a solid $S$, bounded by parallel planes perpendicular to the $x$-axis at $x=a$ and $x=b$, with a cross-sectional area perpendicular to the $x$-axis as $A(x)$ for all $x$ in $[a,b]$, has a volume of 
$$\int_a^b {A(x) \ dx}$$
\noindent 
You can visualize this concept as the area under a curve turned 3D. Instead of adding infinitesimally thin rectangular strips, we are adding infinitesimally thin cross sections. \\
\\
\noindent 
In volume by cross-section problems, you will usually not directly be given the function for the cross-sectional area. Instead, you will be given the following two pieces of information which you then can use to find the function: \\
\\
\noindent 
1. The function(s) that define and bound the base of the solid \\
\\
\noindent 
2. The shape of the cross sections (the cross sections will always be perpendicular to the plane of the base)\\
\\
\noindent 
The most important thing is being able to visualize the solid.\\
\\
\noindent For example, let's visualize the following statement: The base of a solid is the region bounded between $y=\sqrt{x}$, the $x$-axis, and $x=3$ and its cross sections perpendicular to the $x$-axis are semicircles. \\
\\
\noindent 
The following is the base of the solid: 

\begin{figure}[htp]
    \centering
    \includegraphics[width=8.25cm]{Screenshot (519).png}
\end{figure}
\noindent 
The following illustrates the cross sections:
\begin{figure}[htp]
    \centering
    \includegraphics[width=8.25cm]{Screenshot (518).png}
\end{figure}\\
\noindent 
The function that defines the base is used to determine a function for the cross sectional area. For example, in the image above, $y=\sqrt{x}$ is the length of the diameter of the semi-circle. This relationship can then be used to find $A(x)$, the function for the area of the semi-circles. 
\section{Volume by cross sections problems}
\noindent 
For any volume by cross sections problem, it is important that you can visualize both the base of the solid as well as the cross-sections. Like with area between curves problems, a good first step is to sketch the base. 
\begin{tcolorbox}[colback=purple!5!white,colframe=purple!75!black]
\textbf{Problem 2.1} Find the volume of the solid whose base is the region bounded by the $x$-axis, $y$-axis, and the line $y=4-x$, and whose cross-sections perpendicular to the $x$-axis are semicircles. 
\end{tcolorbox}
\noindent 
\textit{Solution to problem 2.1:} We begin with a sketch of the base: 

\begin{figure}[htp]
    \centering
    \includegraphics[width=8cm]{Screenshot (522).png}
\end{figure}\\
\noindent 
The area for a semicircle is $\frac{1}{2}\pi R^2$. The diameter of the semicircle rests between the $x$-axis and $y=4-x$ so it can be represented by $4-x$. We can then derive an area function for the cross sections: 
\begin{align*}
    A(x) &= \frac{1}{2} \pi R^2 \\
         &= \frac{1}{2} \pi \frac{(4-x)^2}{2^2} \\
         &= \frac{16-8x+x^2}{8}
\end{align*}
\noindent 
The limits of integration are 0 and 4. With this, we have all the information needed to set up and solve an integral expression:
\begin{align*}
    \text{Volume} &=\int_a^b {A(x) \ dx} \\
                  &=\int_0^4 {\frac{16-8x+x^2}{8}} \ dx \\
                  &=\left[8x^2-4x^2+\frac{x^3}{3}\right]_0^4 \\
                  &= \frac{8}{3}
\end{align*}
\noindent 
The next solid contains right isosceles triangle cross sections. 
\begin{tcolorbox}[colback=purple!5!white,colframe=purple!75!black]
\textbf{Problem 2.2} Find the volume of the solid whose base is the region bounded by the circle $x^2+y^2=4$ and whose cross sections perpendicular to the $x$-axis are right isosceles triangles with a leg on the base of the solid. 
\end{tcolorbox}
\noindent
\textit{Solution to problem 2.2:} We begin with a sketch of the base: 

\begin{figure}[htp]
    \centering
    \includegraphics[width=8cm]{Screenshot (523).png}
\end{figure}
\newpage
\noindent 
Next, because our cross sections are perpendicular to the $x$-axis, we must have our function as $y$ in terms of $x$: 
$$y=\sqrt{4-x^2}$$
\noindent 
The distance between the curve and the $x$-axis is $\sqrt{4-x^2}$. Since the leg of our triangle is bounded between the curve, it has twice that length. We then have that 
\begin{align*}
    A(x) &=\frac{1}{2}bh \\
         &=\frac{1}{2}(2\sqrt{4-x^2})^2 \\
         &=8-2x^2
\end{align*}
\noindent 
The limits of integration are -2 and 2. The volume integration is then as follows: 
\begin{align*}
    \text{Volume} &=\int_a^b {A(x) \ dx} \\
                  &=\int_{-2}^2 {(8-2x^2)} \ dx \\
                  &=\left[ {8x-\frac{2}{3}x^3}\right]_{-2}^2 \\
                  &=\frac{64}{3}
\end{align*}
\noindent 
As seen, once you can visualize the base and cross sections of the solid, putting everything else together comes quite easily.
\section{Integrating with respect to $y$}
\noindent 
We can also solve volume by cross section problems by integrating with respect to $y$. The proper result is as follows: \\
\\
\noindent 
The volume of a solid $S$, bounded by parallel planes perpendicular to the $y$-axis at $y=c$ and $y=d$, with a cross-sectional area perpendicular to the $y$-axis as $A(y)$ for all $y$ in $[c,d]$, has a volume of
$$\int_c^d{A(y)} \ dy$$
\noindent 
Here, the cross sections are perpendicular to the $y$-axis instead of the $x$-axis. You can think of them as horizontally placed instead of vertically placed. This concept is the exact same to when we integrated with respect to $y$ for area between curves. \\
\\
\noindent 
The process should still be the same: sketch the base and visualize the cross sections. 
\begin{tcolorbox}[colback=purple!5!white,colframe=purple!75!black] 
\textbf{Problem 3.1} Find the volume of the solid whose base is the region bounded between the curves $x=y^2-5y$ and $x=-4$, and whose cross sections perpendicular to the $y$-axis are squares. 
\end{tcolorbox}
\noindent
\textit{Solution to problem 3.1:} We begin with a sketch of the region: 

\begin{figure}[htp]
    \centering
    \includegraphics[width=8cm]{Screenshot (525).png}
\end{figure}\\
\noindent 
Because the square cross sections are perpendicular to the $y$-axis, the side of the square is the horizontal distance between the two curves. This distance is $-4-(y^2-5y)$. We then have that 
\begin{align*}
    A(x) &=(\text{side})^2 \\
         &=(-4-y^2+5y)^2 \\
         &=y^4-10y^3+33y^2-40y+16
\end{align*}
\noindent 
Next, for limits of integration, we set the two functions equal and solve for $y$: 
$$y^2-5y=-4$$
$$(y-1)(y-4)=0$$
$$y=1,4$$
\noindent 
Our volume integration is as follows: 
\begin{align*}
    \text{Volume} &= \int_c^d {A(y) \ dy} \\
                  &= \int_1^4 {(y^4-10y^3+33y^2-40y+16)} \ dy \\
                  &=\left[\frac{y^5}{5}-\frac{5}{2}y^4+11y^3-20y^2+16y\right]_1^4 \\
                  &=\frac{81}{10}
\end{align*}
\section{Recap points}
\begin{itemize}
    \item The volume of a solid can be found by integrating the function that gives the area of its cross sections. You can visualize this as taking the area under a curve but in 3D.
    \item When finding the volume of a solid in an interval $[a,b]$ by integrating with respect to $x$, we use the following expression: 
    $$\int_a^b A(x) \ dx$$
    \noindent 
    where $A(x)$ represents the area of the cross sections as a function of $x$. 
    \item When finding the volume of a solid in a vertical interval $[c,d]$ by integrating with respect to $y$, we use the following expression: 
    $$\int_c^d A(y) \ dy$$
    \noindent 
    where $A(y)$ represents the area of the cross sections as a function of $y$. 
    \item Begin all volume by cross sections problems with a sketch of the solid's base. 
    \item It is very important to visualize both the base and the cross sections; setting up the integral expression should follow easily. 
\end{itemize}
\section{Exercises}\\
\noindent 
\textbf{5.1} Find the volume of the solid whose base is the region bounded by the circle $x^2+y^2=4$ and whose cross sections perpendicular to the $x$-axis are squares.\\
\\
\noindent 
\textbf{5.2} Find the volume of the solid whose base is the region bounded by the circle $x^2+y^2=4$ and whose cross sections perpendicular to the $x$-axis are equilateral triangles.\\
\\
\noindent 
\textbf{5.3} Find the volume of the solid whose base is the region bounded between the curves $y=x$ and $y=x^2$, and whose cross sections perpendicular to the $x$-axis are semicircles. 
\end{document}