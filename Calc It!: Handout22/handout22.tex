\title{Integrating Products of Sines and Cosines}
\author{Jonathan Kong}
\date{}
\documentclass[11pt]{scrartcl}
\usepackage{subfiles}
\usepackage[sexy]{evan}
\usepackage[utf8]{inputenc}
\usepackage{ upgreek }
\usepackage{geometry}
\geometry{%}
  letterpaper,
  lmargin=1.5cm,
  rmargin=1.5cm,
  tmargin=2 cm,
  bmargin=2cm,
  footskip=12pt,
  headheight=13.6pt}
\usepackage{url}
\urlstyle{tt}
\usepackage{float}
\usepackage{verbatim}
\usepackage{amsmath}
\usepackage{tcolorbox}
\usepackage[dvipsnames]{xcolor}
\usepackage{amssymb}\usepackage{dcolumn}
\newcolumntype{2}{D{.}{}{2.0}}
\begin{document}
\maketitle
\noindent 

\section{Method for integrating products of sine and cosine}
\noindent
In this section, we will use an integration method that allows for integrating products of sines and cosines. \\
\\
\noindent 
We will use the following trigonometric identity: 
$$\sin ^2 x + \cos ^2 x=1$$
\noindent 
The integrals we will look at are of the following form: 
$$\int{\sin ^m x \cos ^n x \ dx}$$
\noindent 
where one of $m$ or $n$ is odd and the other even. Our procedure for evaluating these integrals is as follows: 
\begin{enumerate}
    \item Identify whether the exponent of $\cos x$ or $\sin x$ is odd (for the purposes of the list, assume $\cos x$ has the odd exponent). 
    \item Split off a factor of $\cos x$
    \item Apply the Pythagorean identity $\cos ^2 x = 1-\sin ^2 x$ and substitute. 
    \item Make the $u$-substitution $u=\sin x$
\end{enumerate}
\noindent 
If $\sin x$ has the odd exponent, all of the steps are simply mirrored using the opposite trig function.  
\section{Integrating products of sine and cosine problems}
\noindent 
The first three steps simplifies the integral into a product that consists of only one factor of either $\sin x$ or $\cos x$. Then, applying $u$-sub eliminates this single factor and the entire integral can be integrated in terms of $u$. In each of the following integrals, keep in mind which substitutions and separations to make based off the exponents. 
\begin{tcolorbox}[colback=purple!5!white,colframe=purple!75!black]
\textbf{Problem 2.1} Evaluate $\int{\sin ^6 x \cos ^7 x \ dx}$
\end{tcolorbox}
\noindent 
\textit{Solution to Problem 2.1:} We begin by identifying which trig function has an odd exponent. Because $\cos x$ has the odd exponent, we split off a factor of $\cos x$: 
\begin{align*}
    \int{\sin ^6 x \cos ^7 x \ dx} &=\sin ^6 x \cos ^6 x \cos x \ dx 
\end{align*}
\noindent 
We now make the substitution $\cos ^2 x=1- \sin ^2 x$: 
\begin{align*}
     \int{\sin ^6 x \cos ^7 x \ dx} &=\sin ^6 x(1- \sin ^2 x)^3 \cos x \ dx
\end{align*}
\noindent 
Finally, to get rid of the $\cos x$, we apply the $u$-substitution $u=\sin x \Rightarrow du=\cos x \ dx$.The integration is then as follows: 
\begin{align*}
     \int{\sin ^6 x \cos ^7 x \ dx} &=\int{u^6(1-u^2)^3 \ du} \\
     &=\int{(u^6-3u^8+3u^{10}-u^{12}) \ du} \\
     &=\frac{1}{7}u^7-\frac{1}{3}u^9+\frac{3}{11}u^{11}-\frac{1}{13}u^{13} + C \\
     &=\frac{1}{7}\sin ^7 x -\frac{1}{3} \sin ^9 x +\frac{3}{11} \sin ^{11} x -\frac{1}{13} \sin ^{13} x +C
\end{align*}
\begin{tcolorbox}[colback=purple!5!white,colframe=purple!75!black]
\textbf{Problem 2.2} Evaluate $\int \cos ^4 x \sin ^5 x \ dx$
\end{tcolorbox}
\noindent 
\textit{Solution to Problem 2.2:} Because the exponent of $\sin x$ is odd, we split off a factor of $\sin x$ and then apply the Pythagorean identity $\sin ^2 x=1- \cos ^2 x$: 
\begin{align*}
    \int{\cos ^4 x \sin ^5 x} \ dx &=\int {\cos ^4 x \sin ^4 x \sin x \ dx} \\
                                   &=\int{\cos ^4 x (1-\cos ^2 x)^2 \sin x \ dx} 
\end{align*}
\noindent 
From here, we make the $u$-substitution $u=\cos x \Rightarrow du=-\sin x \ dx \Rightarrow -du=\sin x \ dx$. The integration is then as follows: 
\begin{align*}
     \int{\cos ^4 x \sin ^5 x} \ dx &=-\int{u^4(1-u^2)^2 \ du} \\
     &=-\int{(u^4-2u^6+u^8)} \ du \\
     &=-\left(\frac{1}{5}u^5-\frac{2}{7}u^7+\frac{1}{9}u^9\right) +C \\
     &=-\frac{1}{5}\cos ^5 x +\frac{2}{7}\cos ^7 x-\frac{1}{9}\cos ^9 x +C
\end{align*}
\section{Recap points}
\begin{itemize}
    \item The integration method explored in this section involves integrals of the following form:
    $$\int \sin ^m \cos ^n x \ dx$$
    \noindent 
    where one of $m$ or $n$ is odd and the other even.  
    \item The method uses the Pythagorean identity $\cos^2 x=1-\sin ^2 x$ as well as $u$-sub to transform the integral above into being in terms of $u$ and ready to integrate. 
\end{itemize}
\noindent 
\section{Exercises}
\noindent 
\textbf{4.1} Evaluate $\int{\sin ^3 x \cos ^4 x \ dx}$ \\
\\
\noindent 
\textbf{4.2} Evaluate $\int{\cos ^ 6 x \sin ^7 x \ dx}$ \\
\\
\noindent 
\textbf{4.3} Evaluate $\int{\sin ^3 4x \cos 4x \ dx}$ 
\end{document} 