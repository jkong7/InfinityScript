\title{Intregration by U-Substitution}
\author{Jonathan Kong}
\date{}
\documentclass[11pt]{scrartcl}
\usepackage{subfiles}
\usepackage[sexy]{evan}
\usepackage[utf8]{inputenc}
\usepackage{ upgreek }
\usepackage{geometry}
\geometry{%}
  letterpaper,
  lmargin=1.5cm,
  rmargin=1.5cm,
  tmargin=2 cm,
  bmargin=2cm,
  footskip=12pt,
  headheight=13.6pt}
\usepackage{url}
\urlstyle{tt}
\usepackage{float}
\usepackage{verbatim}
\usepackage{amsmath}
\usepackage{tcolorbox}
\usepackage[dvipsnames]{xcolor}
\usepackage{amssymb}\usepackage{dcolumn}
\newcolumntype{2}{D{.}{}{2.0}}
\begin{document}
\maketitle
\noindent 

\section{Reversing the chain rule}
\noindent
In this section, we will use integration by $u$-substitution, or $u$-sub for short, to evaluate integrals. \\
\\
\noindent 
For functions such as $\cos x$ or $e^x$, it's very easy to find their anti-derivatives as they simply follow their derivative counterparts.  \\
\\
\noindent 
But what about an integral like $\int 3x^2 \cos{x^3} \ dx$? There is no specific rule or anti-derivative we know of that can be used to evaluate this integral. \\
\\
\noindent 
Let's try dealing with the $x^3$ by making the following substitution: 
$$u=x^3$$ 
\noindent 
Our integral is now 
$$\int 3x^2 \cos u \ dx$$
\noindent 
However, the presence of two different variables in the integral is a problem. We must find a way to eliminate the $3x^2$ and $dx$ and express them as $u$ terms in order to integrate in terms of $u$.\\
\\
\noindent 
We take one more step to do this. Note that 
$$\frac{d}{dx}x^3=3x^2$$
\noindent 
We already made the substitution $u=x^3$ so 
$$\frac{d}{dx}u=3x^2$$
\noindent 
Finally, we move the $dx$ to the other side: 
$$du=3x^2 \ dx$$
Making this substitution, our integral becomes
$$\int \cos u \ du$$
\noindent 
Because the integral contains only one variable, we can integrate as we usually would: 
$$\int \cos u=\sin u +C$$
\noindent 
The last step is to substitute $u=x^3$ back in: 
$$\sin x^3 + C$$
\noindent 
We just evaluated an integral using $u$-sub. \\
\\
\noindent 
Note that this method primarily contained two steps. The first was setting $u$ to an $x$ term, and the second was deriving $du$ from differentiating that $x$ term. Substituting both of these values gave us an integral purely in terms of $u$. \\
\\
\noindent 
We use $u$-sub when dealing with integrals of the following form:
$$\int f[g(x)]g'(x) \ dx$$
\noindent 
By setting $u=g(x)$, we get that $\frac{d}{dx}u=g'(x) \Rightarrow du=g'(x)
 \ dx$. Then, 
 \begin{align*}
\int f[g(x)]g'(x) \ dx &= \int{f(u) \ du} \\
                       &= F(g(x)) + C
 \end{align*}
 \noindent 
where $F'=f$. \\
\\
\noindent
Note that this is just the chain rule for derivatives but in reverse.\\
\\
\noindent 
To decide what to set $u$ as, find the portion of the integral whose derivative matches some other portion in the integral. This way, all parts of the integral can be eliminated and replaced with $u$. 
\section{Integration by $u$-substitution problems}
We will now evaluate various integrals using $u$-sub. The most important part is correctly identifying which portion of the integral to set to $u$. 
\begin{tcolorbox}[colback=purple!5!white,colframe=purple!75!black]
\textbf{Problem 2.1} Evaluate the following integrals: \\
\\
\noindent 
(a) $\int 2xe^{x^2} \ dx$\\
\\
\noindent 
(b) $\int (8x-8)(2x^2-4x)^4 \ dx$ \\
\\
\noindent 
(c) $\int \frac{1}{4} \sin x \cos x \ dx$
\end{tcolorbox}
\noindent 
\textit{Solution to Problem 2.1:} \\
\\
\noindent 
(a) The derivative of $x^2$ is $2x$. This is a clear indication to have $u=x^2$. Differentiating this equation yields $du=2x \ dx$. Substituting these terms yields an integral in terms of $u$ and so we can now integrate: 
\begin{align*}
    \int 2xe^{x^2} \ dx &=\int e^u \ du \\
                        &=e^u+C \\
                        &=e^{x^2}+C
\end{align*} 
(b) The derivative of $2x^2-4x$ is $4x-4$ which can clearly be used to eliminate the other portion of the integral. We thus have $u=2x^2-4x \Rightarrow du=(4x-4) \ dx$. From here, we note that $2 \du=(8x-8) \ dx$. We substitute these terms and integrate: 
\begin{align*}
\int (8x-8)(2x^2-4x)^4 &= \int{2u^4 \ du} \\
                       &= \frac{2u^5}{5}+C \\
                       &=\frac{2(2x^2-4x)^5}{5}+C
\end{align*}
(c) We can have either $u=\sin x$ or $u=\cos x$ as both would allow us to eliminate the other trigonometric function. We will go with $u= \sin x$ as we won't have to work with a negative sign. $u= \sin x \Rightarrow du=\cos x \ dx$. We substitute these terms and integrate: 
\begin{align*}
    \int \frac{1}{4}\sin x \cos x \ dx &=\int \frac{1}{4} u \du \\
                                       &= \frac{1}{4}\left(\frac{u^2}{2}\right)+C\\
                                       &=\frac{1}{8}\sin ^2 x +C
\end{align*}


\section{Harder $u$-substitution problems}
The next two integrals are not so cut and dry. It will take more than just the derivative of the $u$-portion to eliminate the other portion.
\begin{tcolorbox}[colback=purple!5!white,colframe=purple!75!black]
\textbf{Problem 3.1} Evaluate $\int (x+1)(x-2)^{10}\ dx$ 
\end{tcolorbox} 
\noindent 
\textit{Solution to Problem 3.1:} We begin with $u=x-2 \Rightarrow du=dx$. Substituting these terms alone do not eliminate the $x+1$ portion of the integral. Therefore, we make a second $u$-sub by noting that since $u=x-2$ $\Rightarrow$ $x+1=u+3$. Substituting all terms will now yield an integral with only $u$ terms and we can integrate:
\begin{align*}
    \int (x+1)(x-2)^{10}\ dx &= \int (u+3)u^{10} \ du \\
                             &= \int (u^{11}+3u^{10}) \ du \\
                             &= \frac{u^{12}}{12}+\frac{3u^{11}}{11}+C \\
                             &=\frac{(x-2)^{12}}{12}+\frac{3(x-2)^{11}}{11}+C
\end{align*}
\noindent 
This method enables $u$-sub to be used in a large variety of integrals. At first, this integral may not have seemed solvable using $u$-sub as the highest power in the two portions are equal; the derivative of the $u$-portion would therefore not cancel out the other portion. But this problem was solved when we made one extra substitution.
\begin{tcolorbox}[colback=purple!5!white,colframe=purple!75!black]
\textbf{Problem 3.2} Evaluate $\int \frac{x^3+1}{x+2} \ dx$
\end{tcolorbox}
\noindent 
\textit{Solution to Problem 3.2:} It may seem intuitive to have $u=x^3+1$ as this is the portion with the highest power. However, by doing this, there is no way to eliminate the $x+2$. Therefore, we begin with $u=x+2 \Rightarrow du=dx$. From here, we have that $u=x+2 \Rightarrow x=u-2 \Rightarrow x^3+1=(u-2)^3+1$. Our integration is then as follows: 
\begin{align*}
    \int \frac{x^3+1}{x+2} \ dx &=\int \frac{(u-2)^3+1}{u} \ du \\
                                &=\int \frac{u^3-6u^2+12u-8+1}{u} \ du \\
                                &=\int \frac{u^3-6u^2+12u-7}{u} \ du \\
                                &=\int (u^2-6u+12-\frac{7}{u}) \ du \\
                                &=\frac{u^3}{3}-3u^2+12u-7\ln \lvert u \rvert +C \\
                                &=\frac{(x+2)^3}{3}-3(x+2)^2+12(x+2)-7 \lvert x+2 \rvert +C
\end{align*}
\section{Recap points}
\begin{itemize}
    \item Integration by $u$-sub is primarily used with integrals of the following form: 
    $$\int f[g(x)]g'(x) \ dx$$
    \noindent 
    Setting $u=g(x)$ allows the entire integral to be in terms of just $u$. 
    \item You can think of $u$-sub as reversing the chain rule for derivatives. 
    \item Set $u$ to be the portion of the integral whose derivative matches the other portion. 
    \item $u$-sub can also be used in more unsuspecting problems. When the derivative of the $u$-portion is not enough to eliminate the other portion, try making an extra substitution by moving terms around. 
\end{itemize}
\section{Exercises}\\
\\
\noindent 
\textbf{5.1} Evaluate $\int{(3x^2+2)(x^3+2x)^5} \ dx$\\
\\
\noindent 
\textbf{5.2} Evaluate $\int{(4x)e^{x^2+5}} \ dx$\\
\\
\noindent 
\textbf{5.3} Evaluate $\int \frac{\cos{\ln x}}{x} \ dx$\\
\\
\noindent 
\textbf{5.4} Evaluate $\int \frac{4x^3-3}{\sqrt{x^4-3x+5}} \ dx$\\
\\
\noindent 
\textbf{5.5} Evaluate $\int x\sqrt{2+x} \ dx$
\end{document}
