\title{Newton’s method}
\author{Jonathan Kong}
\date{}
\documentclass[11pt]{scrartcl}
\usepackage{subfiles}
\usepackage[sexy]{evan}
\usepackage[utf8]{inputenc}
\usepackage{ upgreek }
\usepackage{geometry}
\geometry{%}
  letterpaper,
  lmargin=1.5cm,
  rmargin=1.5cm,
  tmargin=2 cm,
  bmargin=2cm,
  footskip=12pt,
  headheight=13.6pt}
\usepackage{url}
\urlstyle{tt}
\usepackage{float}
\usepackage{verbatim}
\usepackage[margin=1in]{geometry}
\usepackage{amsmath}
\usepackage{tcolorbox}
\usepackage[dvipsnames]{xcolor}
\usepackage{amssymb}\usepackage{dcolumn}
\newcolumntype{2}{D{.}{}{2.0}}
\begin{document}
\maketitle
\noindent

\section{Newton's method: Concept and derivation}
\noindent
In this section, we will use Newton's method to estimate the roots of functions. In algebra, students learn to find solutions to first-degree functions. As the subjects progress, second-degree and third-degree functions are introduced, and subsequently, formulas for their roots are learned. \\
\\
\noindent 
However, what happens when the degree of a function is too large for a formula of its roots? Mathematician Niels Henrik Abel proved that it is impossible to construct a simple formula for the solutions of a fifth-degree polynomial or higher.\\
\\
\noindent 
A way to approximate the real roots of polynomials exists with Newton's method. Take a look at the image below:\\
\\
\noindent 
\begin{figure}[htp]
    \centering
    \includegraphics[width=12cm]{Screenshot (478).png}
\end{figure}\\
\noindent
Newton's method uses a series of linear approximations to get closer to the root of a function $r$. In the figure above, we try to find the root $r$ of the equation $f(x)=0$. Our first step is to make a rough estimate, which we will denote as $x_0$. If $f(x_0)=0$, we have found the root of the function. If $f(x_0)$ does not equal zero, we take the linear approximation at $x_0$ and find its $x$-intercept $x_1$. Similarly, if $f(x_1)$ does not equal zero, we take the linear approximation at $x_1$ and find its $x$-intercept $x_2$. Each subsequent $x$-intercept value gets closer and closer to the actual root. This is the general idea behind Newton's method. \\
\\
\noindent
Let's derive a formula to implement Newton's method analytically. The equation of the linear approximation for a function $f$ at a point $x_0$ is 
$$y-f(x_0)=f'(x_0)(x-x_0)$$
This line crosses the $x$-axis at some point $(x_1,0)$. Substituting this point into the equation, we get
\begin{center}
0 - $f(x\textsubscript0)$ = $f'(x\textsubscript0)(x_1-x_0)$
\end{center}
and solving for $x_1$:
$$x\textsubscript1 = x\textsubscript0 - \frac{f(x\textsubscript0)}{f'(x\textsubscript0)}$$
We follow this pattern: every time a value is not equal to zero, we substitute the next number to find a new $x$. This leads us to the general form of Newton's method: 
$$x\textsubscript{n+1} = x\textsubscript{n} - \frac{f(x\textsubscript{n})}{f'(x\textsubscript{n})}$$
Every subsequent substitution gets us closer and closer to the actual root.  
\section{Newton's method problems} 
\noindent 
For the remainder of the section, use of a calculator or some other computing device may be necessary. 
\begin{tcolorbox}
[colback=purple!5!white,colframe=purple!75!black]
\textbf{Problem 2.1} Use Newton's method to approximate the real root of $f(x)=x^3 - x - 1 = 0$ to five decimal places. 
\end{tcolorbox}
\noindent 
\textit{Solution to problem 2.1:} 
\noindent 
The first step is to make an approximation of where we think our root is. We note that $f(1)=-1$ and $f(2)=5$ so our root must lie between 1 and 2; $x_0=1.5$ serves as a good estimate. \\
\\
\noindent 
Next, we find the derivative of $f$. $f(x)$ = $x^3 - x - 1$ and so $f'(x)=3x^2-1$. The Newton's method formula is then 
$$x_{n+1}=x_n-\frac{x^3-x-1}{3x^2-1}$$
Substituting in 1.5 for $x_0$ yields 
\begin{align*}
    x_1 =1.5-\frac{1.5^3-1.5-1}{3(1.5)^2-1} \approx 1.3478261
\end{align*} 
Substituting in 1.3478261 for $x_1$ yields 
\begin{align*}
    x_2 =1.3478261-\frac{1.3478261^3-1.3478261-1}{3(1.3478261)^2-1} \approx 1.3252004
\end{align*}
Substituting in 1.3252004 for $x_2$ yields 
\begin{align*}
    x_3 =1.3252004-\frac{1.3252004^3-1.3252004-1}{3(1.3252004)^2-1} \approx 1.3247182
\end{align*}
Finally, substituting in 1.3247182 for $x_3$ yields 
\begin{align*}
    x_4 =1.3247182-\frac{1.3247182^3-1.3247182-1}{3(1.3247182)^2-1} \approx 1.3247180
\end{align*}
\noindent 
Because the first five decimals have been repeated, we can stop. Newton's method gives that the root, to five decimal places, is 1.32471.\\
\\
\noindent 
There are some issues with Newton's method that should be known. \\
\\
\noindent 
If the derivative of a function is equal to 0 when you plug in some $x$ value, you are dealing with a horizontal tangent. Horizontal tangents never cross the $x$-axis so the next approximation will never occur. You will have to use a different value and try again.\\
\\
\noindent 
Also, Newton's method sometimes fails to converge to a root. For example, the function $f(x)=x^{1/3}$ has one root at $x=0$. Newton's method results in
$$x\textsubscript{n+1} = x\textsubscript{n} - \frac{(x\textsubscript{n})\textsuperscript{1/3}}{\frac{1}{3}(x\textsubscript{n})\textsuperscript{-2/3}}$$
or $$x\textsubscript{n+1} = -2x\textsubscript{n}$$
If we try $x_0=1$, our values get progressively farther away the actual root: -2 to 4 to -8.
\section{Recap points}
\begin{itemize}
    \item Newton's method is used to estimate the roots of functions. 
    \item It achieves this through using a series of linear approximations. First, an estimate for the value of the root is made: $x_0$. The $x$-intercept of the linear approximation line at $x_0$ is found: $x_1$. Then, the $x$-intercept of the linear approximation line at $x_1$ is found: $x_2$. This process of finding successive $x$-intercepts gets you closer and closer to the actual root. 
    \item The general form of Newton's method is as follows: 
    $$x\textsubscript{n+1} = x\textsubscript{n} - \frac{f(x\textsubscript{n})}{f'(x\textsubscript{n})}$$
    \item There are some cases where Newton's method does not work. Common ones include when the derivative value is zero or when the successive approximations fails to converge to a root. 
\end{itemize}
\section{Exercises}\\
\\
\noindent 
\textbf{4.1} Approximate $\sqrt{2}$ to four decimal places by applying Newton's Method to the equation $x^2-2=0$.\\
\\
\noindent 
\textbf{4.2} Approximate $\sqrt{5}$ to four decimal places by applying Newton's Method to the equation $x^2-5=0$. \\
\\
\noindent 
\textbf{4.3} Use Newton's method to approximate the real root of $f(x)=x^3-2x-2$ to four decimal places. 
\end{document}