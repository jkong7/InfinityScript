\title{Implicit Differentiation}
\author{Jonathan Kong}
\date{}
\documentclass[11pt]{scrartcl}
\usepackage{subfiles}
\usepackage[sexy]{evan}
\usepackage[utf8]{inputenc}
\usepackage{ upgreek }
\usepackage{geometry}
\geometry{%}
  letterpaper,
  lmargin=1.5cm,
  rmargin=1.5cm,
  tmargin=2 cm,
  bmargin=2cm,
  footskip=12pt,
  headheight=13.6pt}
\usepackage{url}
\urlstyle{tt}
\usepackage{float}
\usepackage{verbatim}
\usepackage[margin=1in]{geometry}
\usepackage{amsmath}
\usepackage{tcolorbox}
\usepackage[dvipsnames]{xcolor}
\usepackage{amssymb}\usepackage{dcolumn}
\newcolumntype{2}{D{.}{}{2.0}}
\begin{document}
\maketitle
\noindent

\section{Implicit and explicit functions}
\noindent
We can often classify functions as either explicit or implicit. An explicit function is a function in which the dependent variable is expressed in terms of the independent variable. The functions that we have been dealing with so far have mostly been all explicit. Some examples are
$$y=x^2+5$$
$$y=\sin x+2\tan x$$
In both cases, $y$ is given \textit{explicitly} as a function of $x$. \\
\noindent\\
The easiest way to recognize implicit functions is by looking at whether or not the dependent and independent variables are mixed. One of the simplest examples is:
$$x^2+y^2=25$$
Here, $y$ is not explicitly defined as a function of $x$ as it is not isolated on one side of the equation. We can make it explicit by isolating $y$:
$$y=\pm\sqrt {25-x^2}$$
Some more examples of implicit functions are:
$$xy^2+12x=4y$$
$$\sin x +2y=12$$
Again, $y$ is not explicitly expressed as a function of $x$. 
\section{Implicit differentiation}
Implicit differentiation, just as its name suggests, is the technique for differentiating implicit functions. In the example of $x^2+y^2=25$, we could first isolate $y$ and then differentiate the function. However, this yields a lot more unnecessary and long steps. In more complicated implicit functions, the process of isolating $y$ and then differentiating will become longer and even close to impossible. \\
\noindent\\
To take the derivative of $x^2+y^2=25$, we keep it how it is and differentiate both sides with respect to $x$, treating $y$ as a function of $x$:
\begin{align*}
    \frac{d}{dx}(x^2+y^2)=\frac{d}{dx}(25)\\
    \frac{d}{dx}(x^2)+\frac{d}{dx}(y^2)=0
\end{align*}
By the power property, $\frac{d}{dx}(x^2)=2x$. By the same logic, we may think that $\frac{d}{dx}(y^2)=2y$. However, we have to keep in mind that $y$ is a function of $x$. What this means is that we can think of $y$ as $f(x)$. $\frac{d}{dx}(f(x))^2$ is the derivative of a composite function, where $f(x)$ is the inner function and the square function is the outer. Since we are dealing with a composite function, the chain rule gives: 
$$\frac{d}{dx}(f(x))^2=2f(x)f'(x)=2yy'$$
The derivative of $y$ with respect to $x$ can be denoted as $y'$ or $\frac{dy}{dx}$. For simplicity sake, we will use $y'$ in this section.\\
\noindent\\
We have that 
$$2x+2yy'=0$$
$$y'=-\frac{x}{y}$$\\
\noindent
Let's take a look at one more example: Find $y'$ for $y=xy^2+x^3y^2$.\\
$$\frac{d}{dx}y=\frac{d}{dx}(xy^2+x^3y^2)$$
$$y'=\frac{d}{dx}(xy^2)+\frac{d}{dx}(x^3y^2)$$
We can use the product property to expand out the right side:
$$y'=(x)'(y^2)+(x)(y^2)'+(x^3)'(y^2)+(x^3)(y^2)'$$
Again, we use the chain rule for the $y$ terms. $(y^2)'=2yy'$.
$$y'=y^2+2xyy'+3x^2y^2+2x^3yy'$$
To isolate $y'$, we move all the terms with it to one side:
$$y'-2xyy'-2x^3yy'=y^2+3x^2y^2$$
$$y'(1-2xy-2x^3y)=y^2+3x^2y^2$$
$$y'=\frac{y^2+3x^2y^2}{1-2xy-2x^3y}$$\\
\begin{tcolorbox}
[colback=purple!5!white,colframe=purple!75!black]
\textbf{Problem 2.1} Find the equation of the line tangent to $(x-y)^2=x+y-1$ at $\left(\frac{1}{2}, \frac{1}{2}\right)$.
\end{tcolorbox}
\noindent
\textit{Solution to Problem 2.1:} The slope of the tangent line is the derivative of $y'$ at  $\left(\frac{1}{2}, \frac{1}{2}\right)$. 
$$\frac{d}{dx}(x-y)^2=\frac{d}{dx}(x)+\frac{d}{dx}(y)+\frac{d}{dx}(-1)$$
We can use the chain rule on the left side:
$$2(x-y)(x-y)'=1+y'$$
$$2(x-y)(1-y')=1+y'$$
We could keep going and eventually isolate $y'$ but it is much easier to just plug in $\left(\frac{1}{2}, \frac{1}{2}\right)$ for $x$ and $y$ here:
$$2(0)(1-y')=1+y'$$
$$y'=-1$$
The tangent line has slope $-1$ and passes through the point $\left(\frac{1}{2}, \frac{1}{2}\right)$:
$$y=-x+1$$
\begin{tcolorbox}
[colback=purple!5!white,colframe=purple!75!black]
\textbf{Problem 2.2} Find $y'$ for the following:\\
\noindent\\
(a) \;\;\;\;$x^3+y^3=4$\\
\noindent\\
(b) \;\;\;\;$y=\sin (x+y)$\\
\noindent\\
(c) \;\;\;\;$12xy+y^2=2xy^2$
\end{tcolorbox}
\noindent
\textit{Solution to Problem 2.2:}\\
\noindent\\
(a) Differentiating both sides:
\begin{align*}
    \frac{d}{dx}(x^3)+\frac{d}{dx}(y^3)=0
\end{align*}
By the chain rule, $\frac{d}{dx}(y^3)=\frac{d}{dx}(f(x))^3=3f(x)^2f'(x)=3y^2y'$.
$$3x^2+3y^2y'=0$$
$$y'=-\frac{x^2}{y^2}$$\\
\noindent\\
(b) Differentiating both sides:
$$\frac{d}{dx}(y)=\frac{d}{dx}(\sin (x+y))$$
We can differentiate the right by using the chain rule where sin is the outer function and $(x+y)$ is the inner:
$$y'=\cos (x+y)(x+y)'$$
$$y'=\cos(x+y)(1+y')$$
Expanding and isolating:
$$y'=\cos (x+y)+y'\cos (x+y)$$
$$y'-y'\cos (x+y)=\cos (x+y)$$
$$y'(1-\cos (x+y))=\cos (x+y)$$
$$y'=\frac{\cos (x+y)}{1-\cos (x+y)}$$\\
\noindent\\
(c) Differentiating both sides:
$$\frac{d}{dx}(12xy)+\frac{d}{dx}(y^2)=\frac{d}{dx}(2xy^2)$$
$$12[(x)'(y)+(x)(y)']+2yy'=2[(x)'(y^2)+(x)(y^2)']$$
$$12(y+xy')+2yy'=2(y^2+2xyy')$$
Expanding and isolating:
$$12y+12xy'+2yy'=2y^2+4xyy'$$
$$12xy'+2yy'-4xyy'=2y^2-12y$$
$$y'(12x+2y-4xy)=2y^2-12y$$
\begin{align*}
    y' & = \frac{2y^2-12y}{12x+2y-4xy}\\
       & = \frac{y^2-6y}{6x+y-2xy}
\end{align*}
\section{Recap points}
\begin{itemize}
    \item An implicit function is one where the dependent and independent variables are mixed. 
    \item Implicit differentiation is the technique for differentiating implicit functions. 
    \item We start by differentiating both sides of the equation, keeping in mind that $y$ is a function of x. The chain rule is often used when dealing with $y$-terms. 
    \item Then, collect all the $y'$ terms and isolate. 
\end{itemize}
\section{Exercises}\\
\noindent
\textbf{4.1} Find $y'$ for the following:\\
\noindent\\
(a) $x^3+2xy=10$\\
\noindent\\
(b) $xy^2+y^3=12x$\\
\noindent\\
(c) $\cos x+\sin (y+2)=x$\\
\noindent\\
(d) $2x+y=x(y^2+x^2)$\\
\noindent\\
\textbf{4.2} Find the equation of the line tangent to $(x^2+y^2)^3=8x^2y^2$ at $(-1,1)$.


\end{document}
