\documentclass[11pt]{scrartcl}
\usepackage[utf8]{inputenc}
\usepackage{sectsty}
\usepackage{graphicx}
\usepackage{asymptote}
\usepackage{tikz}
\usepackage{tcolorbox}
\usepackage{amsmath}
\usepackage{mathtools}
\usepackage{physics}
\usepackage{textcomp}
\usepackage{siunitx}
\usepackage{dirtytalk}
\usepackage[autostyle]{csquotes}


\DeclareMathOperator{\min}{min}


\makeatletter
\renewcommand\section{\@startsection{section}{1}{\z@}%
                                   {-3.5ex \@plus -1ex \@minus -.2ex}%
                                   {2.3ex \@plus.2ex}%
                                   {\normalfont\large\bfseries}}
\makeatother
\title{\normalfont\notesize\textbf{Chapter 7}}
\author{Jonathan Kong}
\date{}

\begin{document}
\maketitle
\section{Dynamic Recursion}
In this chapter, we will be looking at recursion, a counting technique that allows us to greatly simplify otherwise extremely lengthy problems. Recursion is a method in where we express each value in a sequence in terms of previous values. For example, we can define the sequence 
$$2, 7, 13, 18,...$$
by using the recursive equation 
$${a_n}={a_{n-1}} +5$$
where $a_n$ is the nth term in the sequence and ${a_0}=2$. \\
\\
\noindent
Perhaps the most well known recursion is the Fibonacci sequence
$$1,1,2,3,5,8,13,21,34,...$$
which can be defined using the recursive equation 
$$F_{n}=F_{n-1}+F_{n-2}$$
where $F_n$ is the nth term in the sequence and $F_{0}=0$ and $F_{1}=1$. \\
\\
\noindent
Let's look at how recursion can be applied to counting problems by examining a simple sequence-forming problem. 
\\
\begin{tcolorbox}
\textbf{Problem 7.1} Let $s_n$ be the number of binary sequences of length $n$ with no consecutive 0s (a binary sequence contains only 0s and 1s). \\
(a) If the first digit is 1, how many ways are there to construct the remaining digits?\\
(b) If the first digit is 0, how many ways are there to construct the remaining digits? \\
(c) Find a recursive formula for $s_n$. \\
(d) How many different binary sequences of length 8 with no consecutive 0's are there?
\end{tcolorbox}
\noindent 
We can proceed by considering the first digit of the sequence. If the first digit is 1, there are no restrictions on the next digit. In other words, the next digit can be a 0 or a 1. Therefore the number of ways to construct the remaining $n-1$ digits is $s_{n-1}$. 
$$1{\underbrace{.....}_{s_{n-1}}}$$


If the first digit is 0, the next digit can only be a 1 as our sequence cannot have consecutive 0's. The third digit can then be a 0 or 1, so the number of ways to construct the remaining $n-2$ digits is $s_{n-2}$. 
$$01{\underbrace{.....}_{s_{n-2}}}$$
Therefore, we have the recursive equation ${s_n}={s_{n-1}}+s_{n-2}$, for all $n > 2$. 
$$s_n=1{\underbrace{.....}_{s_{n-1}}}+01{\underbrace{.....}_{s_{n-2}}}$$
The number of sequences of length 8 is $s_8$. First note that ${s_1}=2$ and ${s_2}=3$. There are 2 sequences of length 1: 0, 1. There are 3 sequences of length 2: 01, 10, 11. Starting with this and using our recursive equation, we can compute $s_8$:
$${s_3}={s_2}+{s_1}=3+2=5$$
$${s_4}={s_3}+{s_2}=5+3=8$$
$$...$$
$${s_8}={s_7}+{s_6}=34+21=55$$\\
\noindent
In this problem, we were able to generate a recursive equation by breaking apart our initial situation into two cases. Breaking apart a situation proposed by a problem into cases that allow for a recurrence relation to be found is the basis of using recursion to solve counting problems.  
\\
\begin{tcolorbox}
\textbf{Problem 7.2} Grant the grasshopper must travel across 11 tiles. If, with each hop, he travels either 1 or 2 tiles, how many different ways can George make his journey?
\end{tcolorbox}
\noindent 
We can find a recursive equation by considering the last step Grant takes. Suppose that Grant travels across $n$ tiles. For his last hop, he either takes a 1-tile hop after travelling $n-1$ tiles or he takes a 2-tile hop after travelling $n-2$ tiles. Let $h_n$ be the number of ways Grant can travel across $n$ tiles. We then have the recursive equation ${h_n}={h_{n-1}}+{h_{n-2}}$, for all $n > 2$. ${h_1}=1$ because we can only take a 1-tile hop and ${h_2}=2$ because we can take two 1-tile hops or one 2-tile hop. Starting with this and using our recursive equation, we can compute $h_{11}$:
$${h_3}={h_2}+{h_1}=2+1=3$$
$${h_4}={h_3}+{h_2}=3+2=5$$
$$...$$
$${h_{11}}={h_{10}}+{h_9}=89+55=144$$


\begin{tcolorbox}
\textbf{Problem 7.3} In a secret society, everything is paid for using three types of coins. A C-coin is worth 5\$, a B-coin is worth 10\$, and an A-coin is worth 15\$. John, a member of the secret society, wants to pay for a new set of headphones that costs 50\$. If he has an infinite amount of each coin, how many different orders are there in which he can give the coins to pay for the headphones? 
\end{tcolorbox}
\noindent 
Let $O_n$ be the number of possible orders in which John can hand his coins. Consider the first coin John gives. If it is a C-coin, he can hand in the remaining coins in $O_{n-5}$ ways. If the first coin is a B-coin, he can hand in the remaining coins in $O_{n-10}$. Finally, if the first coin is an A-coin, he can hand in the remaining coins in $O_{n-15}$ ways. Therefore, we have that ${O_n}=O_{n-5}+O_{n-10}+O_{n-15}$. For the initial cases, we have that ${O_5}=1$, ${O_{10}}=2$, and ${O_{15}}=4$. Starting with this and using our recursive equation, we can compute $O_{50}$:
$${O_{20}}={O_{15}}+{O_{10}}+{O_{5}}=4+2+1=7$$
$${O_{25}}={O_{20}}+{O_{15}}+{O_{10}}=7+4+2=13$$
$$...$$
$${O_{50}}={O_{45}}+{O_{40}}+{O_{35}}=149+81+44=274$$
\noindent
By now, you are probably a lot more familiar with spotting ways to turn a counting problem into a recurrence relation. The next problem is a little more tricky as it contains a few more elements, but can be solved much the same as the previous problems. 
\\
\begin{tcolorbox}
\textbf{Problem 7.4} Grant the grasshopper now returns from his journey using the same 11 tile path. This time, before he makes a 1 tile jump, he marks his location either red or blue and before a 2 tile jump, he marks his location either green or brown. How many different ways can Grant make his journey back?
\end{tcolorbox}
\noindent 
Let $r_n$ be the number of ways Grant can make his return journey across $n$ tiles. Again, let's consider the last hop Grant takes when he travels across $n$ tiles. If he takes a 1-tile hop after travelling $n-1$ tiles, he has two options for the color he lays down. If he takes a 2-tile hop after travelling $n-2$ tiles, he has two options for the color he lays down. In other words, after Grant travels across $n-1$ or $n-2$ tiles, there are 2 ways he can finish. Therefore, we have that ${r_n}=2{r_{n-1}+2{r_{n-2}}}$, for all $n > 2$. ${r_1}=2$ because there are two ways to take a 1-tile hop and ${r_2}=6$ because there are four ways to take two 1-tile hops and two ways to take one 2-tile hop. Starting with this and using our recursive equation, we can compute $r_{11}$:
$${r_3}=2{r_2}+2{r_1}=12+4=16$$
$${r_4}=2{r_3}+2{r_2}=32+12=44$$
$$...$$
$${r_{11}}=2{r_{10}}+2{r_9}= $$


\begin{tcolorbox}
\textbf{Problem 7.5} A \textit{domino} is a 1 $\times$ 2 or 2 $\times$ 1 rectangle. A \textit{domino tiling} of a region of the plane is a way of covering it (and only it) completely by non-overlapping dominoes. How many domino tilings are there of a 2 $\times$ 10 rectangle? (Source: HMMT)
\end{tcolorbox}
\noindent 
Let $d_n$ be the number of ways we can tile a 2 $\times$ $n$ rectangle. Suppose that we start the tiling from the left of the rectangle. If we place a $2 \times 1$ domino, we have a $2 \times (n-1)$ rectangle remaining. The only way to begin our tiling with a $1 \times 2$ domino is if we place two to fill a $2 \times 2$ square. We then have a $2 \times (n-2)$ rectangle remaining. Since these two cases are the only way we can begin the tiling, we have that ${d_n}={d_{n-1}}+{d_{n-2}}$, for all $n>2$. There is one way to tile a $2 \times 1$ rectangle and two way to tile a $2 \times 2$ rectangles so ${d_1}=1$ and ${d_2}=2$. Starting from this and using our recursive equation, we can compute $d_{10}$:
$${d_3}={d_2}+{d_1}=2+1=3$$
$${d_4}={d_3}+{d_2}=3+2=5$$
$$...$$
$${d_{10}}={d_9}+{d_8}=55+34=89$$
For the next problem, try using the rules established to form a recurrence relation between different sized towers. 
\\
\begin{tcolorbox}
\textbf{Problem 7.6} A collection of 8 cubes consists of one cube with edge-length $k$ for each integer $k, 1 \le k \le 8.$ A tower is to be built using all 8 cubes according to the rules:
\begin{itemize}
    \item Any cube may be the bottom cube in the tower.
    \item The cube immediately on top of a cube with edge-length $k$ must have edge-length at most $k+2.$
\end{itemize}
How many different towers can be constructed? (Source: AIME)
\end{tcolorbox}
\noindent 
Let $T_n$ be the number of $n$-block towers that satisfy the requirements. If we want to form an $(n+1)$-tower from an $n$-tower, we examine the block of size $n+1$. From the first requirement, one option is to place this block at the bottom. The second requirement tells us that we can place this block on top of the block with size $n-1$ or the block with size $n$. Therefore, we have three ways to form an $(n+1)$-tower from an $n$-tower. This gives the recursive equation ${T_n}=3{T_{n-1}}$. ${T_2}=2$ because both arrangements of blocks with size 1 and 2 work. We note that after $T_2$, we can apply our recursion because the block with size 3 can be placed in three spots on the 2-block tower. Applying our recursion gives:
$${T_8}=3^6(2)=1458$$


\begin{tcolorbox}
\textbf{Problem 7.7} Consider a ternary sequence of length 9. If we can never have two of the same digit adjacent to each other, how many different sequences are there?   
\end{tcolorbox}
\noindent 
Let $s_n$ be the number of ternary sequences of length $n$ with no two of the same digit adjacent to each other. We see that to add a digit at the end of a sequence of length $n$, there are two digits we can use as there will always be two different digits. Therefore, we have two ways to form a sequence with $n+1$ digits from a sequence with $n$ digits, giving us the recursion ${s_n}=2{s_{n-1}}$. ${s_1}=3$ as we can use any of the three digits to form a sequence of length 1. Applying our recursion gives: 
$${s_9}=2^8(3)=768$$
The last two problems are more complicated versions of the sequence-forming problems we previously looked at in the chapter. \\
\begin{tcolorbox}
\textbf{Problem 7.8} There are $2^{10} = 1024$ possible 10-letter strings in which each letter is either an A or a B. Find the number of such strings that do not have more than 3 adjacent letters that are identical. (Source: AIME)
\end{tcolorbox}
\noindent 
Let $s_n$ be the number of $n$-letter strings with no more than 3 adjacent letters identical. Let's consider the first letter of the string. If it is an A, we can have either one, two, or three of them before a B: AB, AAB, AAAB. The same holds for if the string begins with a B, where we can have either one, two, or three of them before an A: BA, BBA, BBBA. Let ${a_n}$ and ${b_n}$ be the number of $n$-letter strings with no more than 3 adjacent letters identical that begin with an A and B, respectively. We have that ${a_n}={b_{n-1}}+{b_{n-2}}+{b_{n-3}}$ and ${b_n}={a_{n-1}}+{a_{n-2}}+{a_{n-3}}$: 
$${a_n}=A{\underbrace{B.....}_{b_{n-1}}}+AA{\underbrace{B.....}_{b_{n-2}}}+AAA{\underbrace{B.....}_{b_{n-3}}}$$
$${b_n}=B{\underbrace{A.....}_{a_{n-1}}}+BB{\underbrace{A.....}_{a_{n-2}}}+BBB{\underbrace{A.....}_{a_{n-3}}}$$
The sequence begins with either an A or B, so ${s_n}={a_n}+{b_n}$. To make our recursion easier, note that ${a_n}={b_n}$ as they are symmetric to each other. Therefore, we can write ${s_n}=2{a_n}$ and ${a_n}={a_{n-1}}+{a_{n-2}}+{a_{n-3}}$. The initial cases can be computed easily: ${a_1}=1$, ${a_2}=2$, ${a_3}=4$. We can now apply our recursion: 
$${a_4}={a_3}+{a_2}+{a_1}=4+2+1=7$$
$${a_5}={a_4}+{a_3}+{a_2}=7+4+2=13$$
$$...$$
$${a_{10}}={a_9}+{a_8}+{a_7}=149+81+44=274$$
$${s_{10}}=2{a_{10}}=2(274)=548$$


\begin{tcolorbox}
\textbf{Problem 7.9} How many sequences of $0$s and $1$s of length $19$ are there that begin with a $0$, end with a $0$, contain no two consecutive $0$s, and contain no three consecutive $1$s? (Source: AMC)
\end{tcolorbox}
\noindent 
Let $s_n$ be the number of sequences of length $n$ that satisfy the requirements in the problem. Using the requirements, we see that the sequence must begin with either a 010 or 0110. The sequence starts again from the 0. Using this, we have the recursive equation ${s_n}={s_{n-2}}+{s_{n-3}}$, for all $s \ge 3$. 
$${s_n}=01{\underbrace{0.....}_{s_{n-2}}}+011{\underbrace{0.....}_{s_{n-3}}}$$
We see that ${s_3}={s_4}={s_5}=1$ as the only possible sequences are 010, 0110, and 01010, respectively. Starting from this and using our recursive equation, we can compute $s_{19}$:
$${s_6}={s_4}+{s_3}=1+1=2$$
$${s_7}={s_5}+{s_4}=1+1=2$$
$${s_8}={s_6}+{s_5}=2+1=3$$
$$...$$
$${s_{18}}={s_{16}}+{s_{15}}=28+21=49$$
$${s_{19}}={s_{17}}+{s_{16}}=37+28=65$$
In this chapter, we looked at the method of recursion to solve counting problems. In all the problems, we were able to break the situation up into cases that allowed us to come up with a recurrence relation, which then allowed us to easily compute what we were looking for. Without recursion, many of the problems would have required extremely lengthy and complicated solutions. 
\end{document}