\title{Local Linear Approximation}
\author{Jonathan Kong}
\date{}
\documentclass[11pt]{scrartcl}
\usepackage{subfiles}
\usepackage[sexy]{evan}
\usepackage[utf8]{inputenc}
\usepackage{ upgreek }
\usepackage{geometry}
\geometry{%}
  letterpaper,
  lmargin=1.5cm,
  rmargin=1.5cm,
  tmargin=2 cm,
  bmargin=2cm,
  footskip=12pt,
  headheight=13.6pt}
\usepackage{url}
\urlstyle{tt}
\usepackage{float}
\usepackage{verbatim}
\usepackage[margin=1in]{geometry}
\usepackage{amsmath}
\usepackage{tcolorbox}
\usepackage[dvipsnames]{xcolor}
\usepackage{amssymb}\usepackage{dcolumn}
\newcolumntype{2}{D{.}{}{2.0}}
\begin{document}
\maketitle
\noindent

\section{Approximating the value of a function using a tangent line}
\noindent 
In this section, we will use tangent lines and derivatives to approximate the value of functions at certain points. \\
\\
\noindent 
To start, take a look at the image below: 
\\
\begin{figure}[htp]
    \centering
    \includegraphics[width=12cm]{Screenshot (495).png}
\end{figure}\\
\\
\noindent 
We know that the tangent line to $f(x)$ at point $(a, f(a))$, which we will denote as $L(x)$, has a slope of $f'(a)$. Using the point-slope formula, we have that 
$$L(x)=f(a)+f'(a)(x-a)$$
The premise of local linear approximation is to plug in an $x$-value to the tangent line equation to approximate the actual value. However, note that in the image above, only for values of $x$ close to $a$ are the tangent line and the actual function close. This tells us that a local linear approximation is only accurate for nearby values of $x$, hence the name \say{local}. In the following problems, we construct local linear approximations as well as use them to approximate functions values. 
\section{Local linear approximation problems}
\begin{tcolorbox}
[colback=purple!5!white,colframe=purple!75!black]
\textbf{Problem 2.1} Construct the local linear approximation for $f(x)=x^4-5x^3+2x+2$ at $x=4$. 
\end{tcolorbox}
\noindent 
\textit{Solution to problem 2.1:} 
We are asked to find the equation of the line tangent to $f$ at $x=4$. \\
\\
\noindent 
We have that $f(4)=-54$. The slope of the tangent line is given by $f'(4)$: 
$$f'(x)=4x^3-15x^2+2$$
and so 
\begin{align*}
    f'(4) &=4(4)^3-15(4)^2+2 \\
          &=18
\end{align*}
The local linear approximation is then
$$L(x)=-54+18(x-4)$$
\noindent 
We will now use local linear approximation to approximate the value of a function.
\begin{tcolorbox}
[colback=purple!5!white,colframe=purple!75!black]
\textbf{Problem 2.2} Using the local linear approximation for $f(x)=x^3$ at $x=2$, estimate the value of $1.98^3$. 
\end{tcolorbox}
\noindent 
\textit{Solution to problem 2.2:} We first note that $f(2)=8$. Solving for the slope of the tangent line, we get that 
$$f'(x)=3x^2$$
and so 
$$f'(2)=12$$
\noindent 
Our local linear approximation is then 
$$L(x)=8+12(x-2)$$
The value $1.98^3$ is $f(1.98)$, which we can approximate as 
\begin{align*}
    f(1.98) \approx L(1.98) &=8+12(1.98-2) \\
                            &=7.76
\end{align*}
\noindent 
The actual value of $1.98^3$ is 7.7624, so our estimate is extremely close. If we were to estimate a value that was further away from $x=2$, our estimation would likely be less accurate as the tangent line would deviate further and further away from the actual function. \section{Recap points}
\begin{itemize}
    \item The tangent line to a point $x=a$ on a function $f$ can be used to approximate values of $f$ near $a$. This is known as local linear approximation. 
    \item The tangent line, denoted $L(x)$, has the following equation: 
    $$L(x)=f(a)+f'(a)(x-a)$$
    \item Local linear approximations are more accurate for closer values of $x$ relative to $a$. 
\end{itemize}
\section{Exercises}\\
\noindent 
\textbf{4.1} Construct the local linear approximation for $f(x)=3x^2-3x+1$ at $x=1$.\\
\\
\noindent 
\textbf{4.2} Using the local linear approximation for $f(x)=\sqrt x$ at $x=4$, estimate $\sqrt {4.1}$. \\
\\
\noindent 
\textbf{4.3} Using the local linear approximation for $f(x)=x\sin(\pi x^2)$ at $x=2$, estimate the value of $f(2.05)$. 









\end{document}
