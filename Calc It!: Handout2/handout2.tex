\title{Continuity}
\author{Jonathan Kong}
\date{}
\documentclass[11pt]{scrartcl}
\usepackage{subfiles}
\usepackage[sexy]{evan}
\usepackage[utf8]{inputenc}
\usepackage{ upgreek }
\usepackage{geometry}
\geometry{%}
  letterpaper,
  lmargin=1.5cm,
  rmargin=1.5cm,
  tmargin=2 cm,
  bmargin=2cm,
  footskip=12pt,
  headheight=13.6pt}
\usepackage{url}
\urlstyle{tt}
\usepackage{float}
\usepackage{verbatim}
\usepackage[margin=1in]{geometry}
\usepackage{amsmath}
\usepackage{tcolorbox}
\usepackage[dvipsnames]{xcolor}
\usepackage{amssymb}\usepackage{dcolumn}
\newcolumntype{2}{D{.}{}{2.0}}
\begin{document}
\maketitle
\noindent
\section{Defining continuity using limits}   
\noindent
In this section, we will look at continuity, a concept derived directly from limits. \\ 
\\
\noindent 
We can intuitively think of a continuous function as one where we can draw its graph without lifting our drawing tool. For this to occur, it must have no holes or jumps. \\
\\
\noindent 
In the graph below, we see that the function has holes at $x=1$ and $x=5$ and a jump at $x=3$. The function is \textit{discontinuous} at these points. 
\begin{figure}[htp]
    \centering
    \includegraphics[width=8cm]{Screenshot (9).png}
    
\end{figure}

\noindent\\
Just like any other concept however, we must rigorously define continuity. To put the words \enquote{no holes or jumps} into mathematical context, we use limits. \\
\\
\noindent 
For a function to be continuous at a point, the limit to that point has to be equal to the value of the function at that point. This ensures that the function does not have a hole or jump at that point. \\
\\
\noindent 
In the graph above, the value of the function at $x=5$ is not equal to the limit of the function as $x$ approaches 5 and so the point is a discontinuity. The same is true at $x=1$ and $x=3$. Below is the proper definition of continuity at a point: \\
\\
\noindent
Definition: A function $f$ is continuous at a point $a$ within its domain if the following are all true: 
\begin{enumerate}
    \item $f(a)$ is defined 
    \item $\lim_{x \to a} f(x)$ exists
    \item $f(a)=\lim_{x \to a} f(x)$
\end{enumerate} 
\noindent 
A function is called continuous if it is continuous at every point in its domain. Some examples of continuous functions are polynomial and exponential functions. 
\section{Continuity problems}
\noindent 
In each of the following problems, use the three conditions listed above to determine continuity. 


\begin{tcolorbox}
[colback=purple!5!white,colframe=purple!75!black]
\textbf{Problem 2.1} Consider the function 
\[
   f(x)= \begin{dcases}
        x^2+3x & x<0 \\
        \sqrt x +1 & x \geq 0
       
    \end{dcases}
\]
\noindent 
Is the function continuous at $x=0$\:?
\end{tcolorbox}
\noindent
\textit{Solution to Problem 2.1:} Our first step is to determine whether $f(0)$ is defined: 
$$f(0)=\sqrt{0}+1=1$$
\noindent 
Next, we must determine whether the limit to $x=0$ 
exists. To determine if the limit exists, we have to compare the one-sided limits. From the left, we have the following limit: 
$$\lim_{x \to 0^{-}} (x^2+3x)=0$$
From the right, we have the following limit: 
$$\lim_{x \to 0^{+}} (\sqrt x +1)=1$$
Since the one-sided limits differ, the limit does not exist at $x=0$ and so the function is not continuous here. Graphically, since the one-sided limits don't line up, this is a jump discontinuity.\\
\begin{tcolorbox}
[colback=purple!5!white,colframe=purple!75!black]
\textbf{Problem 2.2} Consider the function 
\[
   f(x)= \begin{dcases}
        x+2 & x \neq 3\\
        7 & x = 3
       
    \end{dcases}
\]
\noindent 
Is the function continuous at $x=3$?
\end{tcolorbox}
\noindent
\textit{Solution to Problem 2.2:} Our first step is to determine whether $f(3)$ is defined: 
$f(3)=7$
\noindent 
Next, we must determine whether the limit to $x=3$ exists. The sub-function $y=x+2$ describes the behavior of the function from both sides and so we use it determine the limit: 
$$\lim_{x \to 3} f(x)=\lim_{x \to 3} (x+2)=5$$
We see that 
$$\lim_{x \to 3} f(x)\neq f(3)$$
Because of this, the function is not continuous at $x=3$. \\
\begin{tcolorbox}
[colback=purple!5!white,colframe=purple!75!black]
\textbf{Problem 2.3} Find the value $K$ that would make the function
$$ f(x)= \begin{dcases}
        \frac{2\sin \pi x}{5x} & x \neq 0 \\
         2K & x = 0
       
    \end{dcases}
$$

continuous at $x=0$.
\end{tcolorbox}
\noindent
\textit{Solution to Problem 2.3:} If the function is continuous at $x=0$, then 
$$\lim_{x \to 0} f(x)=f(0)$$
$f(0)$ is equal to $2K$ and 
$$\lim_{x \to 0} f(x)=\lim_{x \to 0} \frac{2\sin \pi x}{5x}=\frac{2}{5}\cdot\lim_{x \to 0} \frac{\sin \pi x}{x}$$
If we have $\pi$ in the denominator of the limit alongside $x$, then the limit is in the form 
$$\lim_{x \to 0} \frac{\sin y}{y}$$
which we know is equal to $1$. We can multiply the limit by $\frac{\pi}{\pi}$ to achieve this form: 
$$\frac{2}{5}\cdot\lim_{x \to 0} \frac{\sin \pi x}{x}=\frac{2 \pi}{5}\cdot \lim_{x \to 0} \frac{\sin \pi x}{\pi x}=\frac{2 \pi}{5}$$
We can now set up and solve the following equation: 
$$2K=\frac{2 \pi}{5}$$
$$K=\frac{\pi}{5}$$
\section{Intermediate value theorem}
One of the most important results that can be derived from continuity is the intermediate value theorem. We will start with a simple and intuitive problem: \\
\begin{tcolorbox}
[colback=purple!5!white,colframe=purple!75!black]
\textbf{Problem 3.1} Consider a continuous function $f$ where $f(1)=5$ and $f(4)=8$.\\
\noindent\\
(a) Must there be a point $c$ between $1$ and $4$ such that $f(c)=7$?\\
\noindent\\
(b) Does the fact that $f$ is continuous need to be necessary for this?
\end{tcolorbox}
\noindent
\textit{Solution to Problem 3.1:}\\
\noindent\\
(a) Draw or visualize any type of continuous function between the two points. It is impossible for the function to go between the two points without intersecting the line $y=7$ somewhere. Because of this, there must be a point $c$ between 1 and 4 such that $f(c)=7$.  \\
\noindent\\
(b) If $f$ is not continuous, there are multiple possibilities such that it does not intersect the line $y=7$. Between the two points, it can have a hole at $y=7$ or have a jump covering the line. Because of this, $f$ must be continuous. \\
\noindent\\
The result of this problem can be generalized into a very important theorem.\\
\noindent\\
Definition of the intermediate value theorem: If $f$ is a continuous function defined on an interval $[a,b]$, where $f(a)\neq f(b)$, and $y$ is a real number between $f(a)$ and $f(b)$, then there exists some $c$ between $a$ and $b$ such that $f(c)=y$. \\
\noindent\\
In the problem above, since $7$ is between $5$ and $8$ and $f$ is continuous, the intermediate value theorem states that there must be a $c$ between 1 and 4 such that $f(c)=7$.\\
\begin{tcolorbox}
[colback=purple!5!white,colframe=purple!75!black]
\textbf{Problem 3.2} Show that the function $f(x)=x^3-3x+1$ has a root that lies in the interval $(0,1)$. 
\end{tcolorbox}
\noindent
\textit{Solution to Problem 3.2:} We must first find the corresponding function values of the ends of the interval:
$$f(0)=1$$
$$f(1)=-1$$
Since the function has no discontinuities on the interval and 0 is between 1 and -1, the intermediate value theorem states that there must be some $c$ in the interval such that $f(c)=0$. This shows that the function has a root on the interval $(0,1)$\\
\noindent\\
The intermediate value theorem is very useful in proving or disproving the existence of certain values in an interval. We can use it in various types of problems. 
\section{Recap points}
\begin{itemize}
    \item A function is continuous if you can draw its graph without lifting your drawing tool. This occurs if it has no holes or jumps. 
    \item For a function $f$ to be continuous at a point $a$, the following conditions must be true:
\begin{enumerate}
    \item $f(a)$ is defined 
    \item $\lim_{x \to a} f(x)$ exists
    \item $f(a)=\lim_{x \to a} f(x)$
\end{enumerate} 
    \item The intermediate value theorem states: if $f$ is a continuous function defined on an interval $[a,b]$, where $f(a)\neq f(b)$, and $y$ is a real number between $f(a)$ and $f(b)$, then there exists some $c$ between $a$ and $b$ such that $f(c)=y$.  
    \item Intuitively, this theorem should make a lot of sense. Between two $y$ values covered by a continuous function, every $y$ value in between is covered. 
\end{itemize}
\section{Exercises}\\
\noindent
\textbf{5.1} Consider the function  
\[
   f(x)= \begin{dcases}
        \frac{x^2-5x+4}{x-4} & x<4 \\
        \sqrt x +1 & x \geq 4
       
    \end{dcases}
\]
Is the function continuous at $x=4$?\\
\noindent\\
\textbf{5.2} Find the values $a$ and $b$ that would make the function
\[
   f(x)= \begin{dcases}
        2ax+2b & x<2 \\
        12 & x = 2\\
        7ax-8b & x>2
       
    \end{dcases}
\]
continuous at $x=2$.\\
\noindent\\
\textbf{5.3} Find the value $K$ that would make the function 
\[
   f(x)= \begin{dcases}
        \frac{3^{x+2}-81}{9^x-81} & x\neq 2 \\
        3K & x = 2\\
    \end{dcases}
\]
continuous at $x=2$.\\
\noindent\\
\textbf{5.4} Show that the function $f(x)=5x^3-2x+4$ has a root that lies in the interval $(-2,0)$. \\
\noindent\\
\textbf{5.5} Show that the function $f(x)=e^x-2\cos x$ has a root that lies in the interval $(0, \pi)$. 
\end{document}
