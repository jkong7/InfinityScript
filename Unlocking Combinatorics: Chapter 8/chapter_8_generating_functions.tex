\documentclass[11pt]{scrartcl}
\usepackage[utf8]{inputenc}
\usepackage{sectsty}
\usepackage{graphicx}
\usepackage{asymptote}
\usepackage{tikz}
\usepackage{tcolorbox}
\usepackage{amsmath}
\usepackage{mathtools}
\usepackage{physics}
\usepackage{textcomp}
\usepackage{siunitx}
\usepackage{dirtytalk}
\usepackage[autostyle]{csquotes}


\DeclareMathOperator{\min}{min}


\makeatletter
\renewcommand\section{\@startsection{section}{1}{\z@}%
                                   {-3.5ex \@plus -1ex \@minus -.2ex}%
                                   {2.3ex \@plus.2ex}%
                                   {\normalfont\large\bfseries}}
\makeatother
\title{\normalfont\notesize\textbf{Chapter 8}}
\author{Jonathan Kong}
\date{}

\begin{document}
\maketitle
\section{Generating Functions}
\noindent 
\begin{tcolorbox}
\textbf{Problem 8.1} The Binomial Theorem states that 
$$(a+b)^n=\sum_{k=0}^n{n \choose k}a^{n-k}b^k$$
\noindent
Using this, find the generating function for the sequence $${{n \choose 0}, {n \choose 1}, {n \choose 2},...,{n \choose n}}$$
\end{tcolorbox}
\noindent 
We have the sequence ${n \choose 0}, {n \choose 1}, {n \choose 2},...,{n \choose n}$, so its corresponding generating function is  
$${n \choose 0} + {n \choose 1}x + {n \choose 2}x^2 +...+ {n \choose n}x^n$$
\noindent 
The Binomial Theorem gives that this is equal to 
$$(x+1)^n$$
\begin{tcolorbox}
\textbf{Problem 8.2} A carnival game offers 3 plush toys, 2 rubber ducks, and 2 water guns as prizes. Assume that all prizes in a category are identical. A winner of the game is allowed to pick 4 prizes. How many ways they do this?\\
(a) Write all the possibilities for number of plush toys the winner can have as a generating function. \\
(b) Do the same for rubber ducks and water guns. \\
(c) Multiply the generating functions and expand. We are looking for the number of ways 4 prizes can be picked. Which term does this correspond to? \\
(d) Use your answer from part (c) to solve the problem. 
\end{tcolorbox}
\noindent 
Either 0, 1, 2, or 3 plush toys are chosen. Since all plush toys are identical, there is only one way to choose a certain number of them. Therefore, we can represent the plush toys with the generating function 
$$1(x^0)+1(x^1)+1(x^2)+1(x^3)=1+x+x^2+x^3$$
where the degree of the variable represents the number of plush toys chosen and the coefficient of that term representing the number of ways to choose that certain amount. \\
\\
\noindent 
Doing the same with rubber ducks and water guns, we find their generating functions to be 
$$1(x^0)+1(x^1)+1(x^2)=1+x+x^2$$
If we multiply the individual generating functions and expand, we get 
\begin{align*}
    (1+x+x^2+x^3)(1+x+x^2)^2 & = (1+x)(1+2x+3x^2+2x^3+x^4)\\
                             & = 1+3x+6x^2+8x^3+8x^4+6x^5+3x^6+x^7
\end{align*}
\begin{tcolorbox}
\textbf{Problem 8.3} Tom wants to select 5 donuts for his friend's birthday. How many ways can he do this if there are 1 vanilla, 2 chocolate, 2 lemon, and 2 strawberry donuts? Assume that donuts of the same type are identical. 
\end{tcolorbox}
\noindent 
The generating function for the vanilla donuts is $1+x$ as there is one way to choose 0 or 1 vanilla donuts. The generating function for the chocolate, lemon, and strawberry donuts is $1+x+x^2$ as there is one way to choose 0, 1, or 2 of each. Multiplying these gives the the generating function for selecting donuts from the combined amount: 
$$(1+x)(1+x+x^2)^3$$
To find the number of ways 5 donuts can be selected, we expand and find the coefficient of the $x^5$ term:
\begin{align*}
    (1+x)(1+x+x^2)^3 & = (1+x)(1+3x+6x^2+7x^3+6x^4+3x^5+x^6)\\
                             & = 1+4x+9x^2+13x^3+13x^4+9x^5+4x^6+x^7
\end{align*}
There are 9 ways to select 5 donuts, as given by the coefficient of the $x^5$ term. \\
\\
\noindent 
To make the algebra easier, we do not need to expand the entire polynomial out. Since we only want the $x^5$ term, we only need to look for the products that will yield an $x^5$ term: 
$$(1)(3x^5)+(x)(6x^4)=9x^5$$
\begin{tcolorbox}
\textbf{Problem 8.4} How many nonnegative integer solutions are there to $x_1+x_2+x_3=6$ if $x_1$ and $x_2$ are odd, $0 \leq {x_1}, {x_2} \leq 6$, and $x_3\leq3$?
\end{tcolorbox}
\noindent 
$x_3$ can be any integer from 0 to 3, so its generating function is $1+x+x^2+x^3$. $x_1$ and $x_2$ can be any odd integer from 0 to 6, so their generating function is $x+x^3+x^5$. The generating function for the values of ${x_1}+{x_2}+{x_3}$ is the product of the individual generating functions: 
$$(1+x+x^2+x^3)(x+x^3+x^5)^2$$
Since we only want the $x^6$ term, we do not need to expand the entire polynomial out; we can instead just look for the products that yield an $x^6$ term: 
$$(1+x+x^2+x^3)(x+x^3+x^5)^2=(1+x+x^2+x^3)(x^2+2x^4+3x^6+2x^8+x^{10})$$
$$(1)(x^6)+(x^2)(2x^4)=3x^6$$
There are 3 solutions, as given by the coefficient of the $x^6$ term. 
\begin{tcolorbox}
\textbf{Problem 8.5} How many ways are there to make $n$ cents using pennies, nickels, dimes, and quarters? Set up but do not solve a generating function for this situation. 
\end{tcolorbox}
\noindent 
We let the cent value of a coin be the degree in our generating function. The generating function of each coin will therefore represent the possible cent values it can cover.\\\
\\
\noindent 
Pennies has the generating function $1+x+x^2+...$\\
\\
\noindent 
Nickels has the generating function $1+x^5+x^{10}+...$\\
\\
\noindent 
Dimes has the generating function $1+x^{10}+x^{20}+...$\\
\\
\noindent 
Quarters has the generating function $1+x^{25}+x^{50}+...$\\
\\
\noindent 
The generating function for the number of ways to make $n$ cents is then 
$$(1+x+x^2+...)(1+x^5+x^{10}+...)(1+x^{10}+1+x^{20}+...)(1+x^{25}+x^{50}+...)$$
and the coefficient of the $x^n$ term yields the answer.
\begin{tcolorbox}
\textbf{Problem 8.6} A street performer currently has 10 spectators around him. At the end of the performance, each spectator will either give the performer \$0 or \$1. How many ways can the performer receive \$7?
\end{tcolorbox}
\noindent 
The generating functions for each of the ten spectators is $1+x$. The generating function for the combined amount of money the spectators give the performer is the product of the individual generating functions: 
$$(1+x)^{10}$$
We want to find the coefficient of the $x^7$ term in the expansion of the expression above. However, expanding the entire expression out would require a lot of algebra. Is there an easier way to get our answer? Recall from Problem 9.1 that the generating function for the sequence $\{{n \choose 0}, {n \choose 1}, {n \choose 2},...,{n \choose n}\}$ is $(1+x)^n$:
$$(1+x)^n={n \choose 0}+{n \choose 1}x+{n \choose 2}x^2+...+{n \choose n}x^n$$
Using this, we see that the coefficient of the $x^7$ term is ${10 \choose 7}=120$, so there are 120 ways the performer can receive \$7.
\begin{tcolorbox}
\textbf{Problem 8.7} Find the number of ways \$15 can be collected from 20 people if 19 of them contribute either \$0 or \$1 and the 20th person contributes \$0, \$1, or \$5.
\end{tcolorbox}
\noindent 
The generating function for each of the first 19 people is $1+x$. The generating function for the 20th person is $1+x+x^5$. The generating function for the combined amount of money is the product of the individual generating functions: 
$$(1+x)^{19}(1+x+x^5)$$
We want to find the coefficient of the $x^{15}$ term in the expansion of the expression above. Each term on the right factor must be multiplied by a term on the left factor to arrive at an $x^{15}$ term. Therefore, we want the $x^{15}$, $x^{14}$, and $x^{10}$ terms from the expansion of $(1+x)^{19}$:
$$(1+x)^{19}={19 \choose 0}+{19 \choose 1}x+{19 \choose 2}x^2+...+{19 \choose 19}x^{19}$$
Therefore, the $x^{15}$ term is given by
$${1}{19 \choose 15}x^{15}+x{19 \choose 14}x^{14}+x^5{19 \choose 10}x^{10}=\left({19 \choose 15}+{19 \choose 14}+{19 \choose 10}\right)x^{15}=107882x^{15}$$
There are 107882 ways for the performer to receive \$15, as given by the coefficient of the $x^{15}$ term.
\\
\begin{tcolorbox}
\textbf{Problem 8.8} Show that $\frac{1}{1-x}$ is the generating function for the sequence $(1,1,1,...)$.
\end{tcolorbox}
\noindent 
Let $A(x)$ be the generating function for this sequence: 
$$A(x)=1+x+x^2+x^3+...$$
We can shift each term over to the right by multiplying $x$ on both sides: 
$$A(x)=1+x+x^2+x^3+...$$
$$xA(x)=x+x^2+x^3+x^4+...$$
We subtract the two equations and solve for $A(x)$: 
$$(1-x)A(x)=1$$
$$A(x)=\frac{1}{1-x}$$
\begin{tcolorbox}
\textbf{Problem 8.9} Using distributions, find the number of nonnegative integer solutions are there to $x_1+x_2+...+x_n=k$? 
\end{tcolorbox}
\noindent 
The number of solutions to this equation is in bijection with the number of ways to distribute $k$ pieces of candy to $n$ children. Therefore, there are ${{n+k-1} \choose {n-1}}$ solutions to the equation. \\
\\
\noindent 
The generating function for each variable is $1+x+x^2+x^3+...=\frac{1}{1-x}$, so the generating function for the sum $x_1+x_2+x_3...+x_n$ is $(1+x+x^2+x^3+...)^n=\frac{1}{(1-x)^n}$. The number of solutions to the equation is therefore given by the coefficient of $x^k$ in the expansion. We solved for the number of solutions using distributions, and therefore we know that the coefficient of $x^k$ in $\frac{1}{(1-x)^n}$ is ${{n+k-1} \choose {n-1}}={{n+k-1} \choose {k}}$\\
\\
\noindent 
This tells us that 
$$\frac{1}{(1-x)^n}={{n-1} \choose {n-1}}+{{n} \choose {n-1}}x+{{n+1} \choose {n-1}}x^2+{{n+2} \choose {n-1}}x^3+...$$


\begin{tcolorbox}
\textbf{Problem 8.10} Find the coefficient of $x^8$ in the following expressions:\\
(a) $\frac{1}{(1-x)^3}$\\
(b) $\frac{x^2}{(1-x)^4}$\\
(b) $(1+x+x^2+x^3+x^4)^3$
\end{tcolorbox}
\noindent 
(a) We have that 
$$\frac{1}{(1-x)^3}={2 \choose 2}+{3 \choose 2}x+{4 \choose 2}x^2+{5 \choose 2}x^3+...$$
The coefficient of $x^8$ is therefore ${10 \choose 2}=45$\\
\\
\noindent
(b) Note that 
\begin{align*}
\frac{x^2}{(1-x)^4} &=x^2 \cdot \frac{1}{(1-x)^4} \\
                    &=x^2 \left[{3 \choose 3}+{4 \choose 3}x+{5 \choose 3}x^2+{6 \choose 3}x^3+...\right]
\end{align*}
Therefore, the coefficient of $x^8$ in $\frac{x^2}{(1-x)^4}$ is given by the coefficient of $x^6$ in $\frac{1}{(1-x)^4}$, which is ${9 \choose 3}=84$ \\
\\
\noindent 
(c) We have that 
$$(1+x+x^2+x^3+x^4)^3=\frac{(1-x^5)^3}{(1-x)^3}$$
We can expand out both the numerator and the denominator 
\begin{align*}
    \frac{(1-x^5)^3}{(1-x)^3} &=(1-3x^5+3x^{10}-x^{15})\left[{2 \choose 2}+{3 \choose 2}x+{4 \choose 2}x^2+{5 \choose 2}x^3+...\right]
\end{align*}
The expansion gives us two terms with an $x^8$: $ 1 \cdot {10 \choose 2}x^8$ and $(-3x^5){5 \choose 2}x^3$. The coefficient of $x^8$ is therefore ${10 \choose 2}-3{5 \choose 2}=15$
\begin{tcolorbox}
\textbf{Problem 8.11}  How many ways are there to roll 3 standard dice so that the sum of the face values is a multiple of 6?
\end{tcolorbox}
\noindent 
The generating function for the sum of the three face values is 
$$(1+x+x^2+x^3+x^4+x^5+x^6)^3=\frac{(1-x^7)^3}{(1-x)^3}$$
We can expand this out
$$(1-3x^7+3x^{14}-x^{21})\left[{2 \choose 2}+{3 \choose 2}x+{4 \choose 2}x^2+{5 \choose 2}x^3+...\right]$$
Note that the greatest possible sum of 3 die is 18 so we are looking at the sums of 6, 12, and 18. \\
\\
\noindent 
The coefficient of $x^6$ is given by the term $1 \cdot {8 \choose 2}x^6$. The coefficient of $x^{12}$ is given by the two terms $1 \cdot {14 \choose 2}x^{12}$ and $(-3x^7){7 \choose 2}x^5$. Lastly, the coefficient of $x^{18}$ is given by the two terms $1 \cdot {20 \choose 2}x^{18}$ and $(3x^{14}){6 \choose 2}x^4$. \\
\\
\noindent 
Therefore, our answer is 
$${8 \choose 2}+{14 \choose 2}-3{7 \choose 2}+{20 \choose 2}+3{8 \choose 2}=330$$
\begin{tcolorbox}
\textbf{Problem 8.12} Find the number of integer solutions to $a+b+c=15$ where $a \ge 7$, $3 \leq b \leq 6$, and $0 \leq c \leq 10$. 
\end{tcolorbox}
\noindent
We begin by writing the generating function for each variable. The generating function for $a$ is 
\begin{align*}
    (x^7+x^8+x^9+...) &= x^7(1+x+x^2+...) \\
                      &=x^7\frac{1}{1-x}
\end{align*}
The generating function for $b$ is 
\begin{align*}
    (x^3+x^4+x^5+x^6) &=x^3(1+x+x^2+x^3) \\
                      &=x^3\frac{1-x^4}{1-x}
\end{align*}
The generating function for $c$ is 
\begin{align*}
    (1+x+x^2+...+x^{10})=\frac{1-x^{11}}{1-x}
\end{align*}
The generating function for $a+b+c$ is therefore 
$$\frac{x^{10}(1-x^4)(1-x^{11})}{(1-x)^3}$$
We expand this out 
$$(x^{10}-x^{14}-x^{21}+x^{25})\left[{2 \choose 2}+{3 \choose 2}x+{4 \choose 2}x^2+...\right]$$
The coefficient of $x^{15}$ is given by the two terms $(x^{10}){7 \choose 2}x^5$ and $(-x^{14}){3 \choose 2}x$\\
\\
\noindent 
Therefore, our answer is 
$${7 \choose 2}-{3 \choose 2}=18$$
\begin{tcolorbox}
\textbf{Problem 8.13} In how many ways can 20 bags of popcorn be distributed to six moviegoers if three will take between 0 and 8, 2 will take between 3 and 7, and one will take any amount?
\end{tcolorbox}
The generating function for each person in the three-group is 
\begin{align*}
    1+x+x^2+...+x^8 &= \frac{1-x^9}{1-x}
\end{align*}
The generating function for each person in the two-group is 
\begin{align*}
    x^3+x^4+x^5+x^6+x^7 &= x^2(1+x^2+x^3+x^4+x^5) \\
                        &= x^2\frac{1-x^6}{1-x}
\end{align*}
The generating function for the single person is
$$1+x+x^2+...=\frac{1}{1-x}$$
The generating function for the total amount of bags taken is therefore 
$$\frac{(1-x^9)^3}{(1-x)^3} \cdot \frac{x^4(1-x^6)^2}{(1-x)^2} \cdot \frac{1}{1-x}=\frac{x^4(1-x^6)^2(1-x^9)^3}{(1-x)^6}$$
Although the expansion of the numerator would be extremely long, we can only focus our attention on the terms in the expansion that would yield our answer. Since we want the coefficient of $x^20$, we can omit the terms in the expansion of the numerator with a degree greater than 20.
$$(x^4-2x^{10}-3x^{13}+x^{16}+6x^{19})\left[{5 \choose 5}+{6 \choose 5}x+{7 \choose 5}x^2+...\right]$$
Therefore, our answer is 
$${21 \choose 5}-2{15 \choose 5}-3{12 \choose 5}+{9 \choose 5}+6{6 \choose 5}=12129$$

\begin{tcolorbox}
\textbf{Problem 8.14} Find the generating function for the recursive sequence 
$$a_{n}-2a_{n-1}-5a_{n-2}=0; a_{0}=1, a_{1}=2$$
\end{tcolorbox}
\noindent 
We begin by writing out the first few terms of the sequence: $(1,2,9,28,101)$ Let $A(x)$ be the generating function for the sequence: 
$$A(x)=1+2x+9x^2+28x^3+101x^4+...$$
We can apply differencing by using the recursive relation to our advantage. We first multiply $A(x)$ by $2x$ and then by $5x^2$ so that we have two new lines. In doing this and subtracting from our original line, all terms of the same degree will be grouped by the recurrence relation: 
$$A(x)=1+2x+9x^2+28x^3+101x^4+...$$
$$2xA(x)=2x+4x^2+18x^3+56x^4+...$$
$$5x^2A(x)=5x^2+10x^3+45x^4+...$$
\begin{align*}
    (1-2x-5x^2)A(x) &=1+(2x-2x)+(9x^2-4x^2-5x^2)+(28x^3-18x^3-10x^3)+...\\
    &=1
\end{align*}
Solving for $A(x)$, we obtain
$$A(x)=\frac{1}{1-2x-5x^2}$$
\begin{tcolorbox}
\textbf{Problem 8.15} The Fibonacci sequence is a series of numbers where each term is the sum of the previous two. The sequence is given by $${f_n}={f_{n-1}}+{f_{n-2}}, {f_0}=0, {{f_1}=1}$$ where $f_n$ is the $nth$ term of the sequence. Let $F(x)$ be the generating function for the Fibonacci sequence 
\end{tcolorbox}
\noindent 
We start with
$$F(x)={f_0}+{f_1}x+{f_2}x^2+{f_3}x^3+...$$
We want to find some manipulation that will allow us to cancel terms out. We can do this by shifting each term once to the right by multiplying by $x$ and twice to the right by multiplying by $x^2$: 
$$F(x)={f_0}+{f_1}x+{f_2}x^2+{f_3}x^3+...$$
$$xF(x)={f_0}x+{f_1}x^2+{f_2}x^3+{f_3}x^4+...$$
$$x^2F(x)={f_0}x^2+{f_1}x^3+{f_2}x^4+{f_3}x^5+...$$
Because ${f_n}={f_{n-1}}+{f_{n-2}}$, we can subtract the last two equations from the first and cancel out a majority of the terms. 
$$F(x)-xF(x)-x^2F(x)=f_0+x(f_1-f_0)+x^2(f_2-f_1-f_0)+x^3(f_3-f_2-f_1)$$
$$(1-x-x^2)F(x)=x$$
$$F(x)=\frac{x}{1-x-x^2}$$
\begin{tcolorbox}
\textbf{Problem 8.16} 
\end{tcolorbox}


\end{document}
